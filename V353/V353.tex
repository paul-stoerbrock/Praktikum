\documentclass[
  captions=tableheading,
  bibliography=totoc, 
  titepage=firstiscover,
]{scrartcl}

\usepackage{blindtext} %neuer input

\usepackage{longtable} % Tabellen über mehrere Seiten

\usepackage[utf8]{inputenc} %neuer input

\usepackage{scrhack}

\usepackage[aux]{rerunfilecheck} %Warnung falls nochmal kompiliert werden muss

\usepackage{fontspec} %Fonteinstellungen

\recalctypearea{}

\usepackage[main=ngerman]{babel} %deutsche Spracheinstellung

\usepackage{ragged2e} %neuer input

\usepackage{amsmath, nccmath}

\usepackage{amssymb} %viele mathe Symbole

\usepackage{mathtools} %Erweiterungen für amsmath

\usepackage{MnSymbol}


\DeclarePairedDelimiter{\abs}{\lvert}{\rvert}
\DeclarePairedDelimiter{\norm}{\lVert}{\rVert}

\DeclarePairedDelimiter{\bra}{\langle}{\rvert}
\DeclarePairedDelimiter{\ket}{\lvert}{\rangle}

\DeclarePairedDelimiterX{\braket}[2]{\langle}{\rangle}{
#1 \delimsize| #2
}

\usepackage{expl3}
\usepackage{xparse}
\NewDocumentCommand \dif {m}
{
\mathinner{\symup{d} #1}
}

\NewDocumentCommand \del {mm}
{
    \mathinner{\frac{\partial #1}{\partial #2}}
}
\NewDocumentCommand \deln {mmm}
{
    \mathinner{\frac{\partial^#3 #1}{\partial #2 ^#3}}
}
\ExplSyntaxOff

\usepackage[
  math-style=ISO,
  bold-style=ISO,
  sans-style=italic,
  nabla=upright,
  partial=upright,
  warnings-off={
    mathtools-colon,
    mathtools-overbracket,
  },
]{unicode-math}

\setmathfont{Latin Modern Math}
\setmathfont{XITS Math}[range={scr, bfscr}]
\setmathfont{XITS Math}[range={cal, bfcal}, StylisticSet=1]

\usepackage[
version=4,
math-greek=default,
text-greek=default,
]{mhchem}

\usepackage[
  locale=DE,
  separate-uncertainty=true,
  per-mode=reciprocal,
  output-decimal-marker={,},
]{siunitx}

\usepackage[autostyle]{csquotes} %richtige Anführungszeichen

\usepackage{xfrac}

\usepackage{float}

\floatplacement{figure}{htbp}

\floatplacement{table}{htbp}

\usepackage[ %floats innerhalb einer section halten
  section,   %floats innerhalb er section halten
  below,     %unterhalb der Section aber auf der selben Seite ist ok
]{placeins}

\usepackage[
  labelfont=bf,
  font=small,
  width=0.9\textwidth,
]{caption}

\usepackage{subcaption} %subfigure, subtable, subref

\usepackage{graphicx}

\usepackage{ulem}

\usepackage{color}

\usepackage{grffile}

\usepackage{booktabs}

\sisetup{separate-uncertainty=true}

\usepackage{microtype} %Verbesserungen am Schriftbild

\usepackage[
backend=biber,
]{biblatex}

\addbibresource{../lit.bib}

\usepackage[ %Hyperlinks im Dokument
  german,
  unicode,
  pdfusetitle,
  pdfcreator={},
  pdfproducer={},
]{hyperref}

\usepackage{bookmark}

\usepackage[shortcuts]{extdash}



\begin{document}
    \title{V353 Relaxationsverhalten des RC-Kreises}
    \author{  
    Tobias Rücker\\
    \texorpdfstring{\href{mailto:tobias.ruecker@tu-dortmund.de}{tobias.ruecker@tu-dortmund.de}
    \and}{,} 
    Paul Störbrock\\
    \texorpdfstring{\href{mailto:paul.stoerbrock@tu-dortmund.de}{paul.stoerbrock@tu-dortmund.de}}{}
    }
    \date{Durchführung: 19.11.2019, Abgabe: 26.11.2019\vspace{-4ex}}
\maketitle
\center{\Large Versuchsgruppe: \textbf{42}}
    
    \begin{abstract}
    \centering
        \textbf{Ziel:} Bestimmung der Zeitkonstante $\tau$ des Relaxationsverhaltens 
    \end{abstract}

\newpage
\tableofcontents
\newpage

% Theorie %%%%%%%%%%%%%%%%%%%%%%%%%%%%%%%%%%%%%%%%%%%%%%%%%%%%%%%%%%%%%%%%%%%%%%%%%%%%%%%%%%%%%%%%%%%%%%%%%%%%%%%%%%%%%%%%%%%%%%%%%%%%%%%%%%%%%%%%%%%

\section{Theorie}\justifying

  Als RC-Kreis wird eine Reihenschaltung bestehend aus einem Kondensator 
  und einem Widerstand beschrieben, durch welchen ein Strom I fließt. 
  Dabei stellt der Kondensator beim Auf- und Entladen einen Relaxationsvorgang dar, 
  was bedeutet, dass die Spannung des Kondensators bei einem Entladevorgang 
  nicht-oszillatorisch zum Anfangszustand zurückkehrt. 
  Beim Entladevorgang eines Kondensators mit der Kapazität C wird eine Gleichspannung $U_0$ 
  angelegt, die nach dem Trennen des Generators über einen Widerstand R abfällt.
  Mit der Anfangsbedingung $U_C(0)=U_0$ kann der Entladevorgang beschrieben werden durch \cite{V353}
  \begin{align}
   U_C(t) &= U_0 \cdot e^{-\frac{1}{\tau} t} \quad &\text{mit} \, \tau &= RC  \label{eq:UcEnt} \text{,}
    \intertext{wobei $\tau$ \;Zeitkonstante des Relaxationsvorgangs genannt wird. Sie bestimmt, 
    wie schnell sich das System seinem Endzustand nähert.
    Beim Aufladevorgang sieht die Gleichung mit der Anfangsbedingung $U(0)=0$ 
    folgendermaßen aus \cite{V353}:}
   U_C(t) &= U_0 \, (1-e^{-\frac{1}{\tau} t}) \quad &\text{mit} \, \tau &= RC  \label{eq:UcAuf}
    \intertext{Relaxationsvorgänge sind beim RC-Kreis allerdings nicht auf den Ent- und 
    Aufladevorgang beschränkt. Auch bei einer Wechselspannung lässt sich dieses 
    Verhalten wiederfinden. Bei einer Wechselspannung stellt sich zwischen Strom 
    und Spannung eine frequenzabhängige Phase $\varphi(\nu)$ \cite{V353}}
  \varphi(\nu) &= \arctan (-\omega \, RC) \quad &\text{mit} \; \omega &= 2 \pi \nu, \label{eq:nu}
  \intertext{ein, die auch durch \cite{V353}}
  \varphi(\nu) &= \frac{\Delta T}{T} \cdot 2 \pi \quad &\text{mit} \; T &= \frac{1}{\nu} \label{eq:nu2}
  \end{align}
  beschrieben werden kann. $\Delta T$\; beschreibt dabei den Abstand der Nulldurchgänge der beiden Spannungen 
  $U_0$ und $U_C(t)$.
  Dadurch wird die Phase zwischen $U_C(t)$ und $U(t)$ bei $\omega \ll \sfrac{1}{\tau}$
  noch ungefähr null sein. Bei steigender Frequenz wird die Phasendifferenz immer größer
  bis sie im Unendlichen gegen den Wert $\sfrac{\pi}{2}$ geht.
  Die Beziehung zwischen der Amplitude, der Spannung und der Kreisfrequenz lautet dann 
  folgendermaßen \cite{V353}:
  \begin{align}
    A(\omega) &= \frac{U_0}{\sqrt{1 + \omega^2 \, R^2  C^2}} \label{eq:A}\\
    \intertext{Oder ergibt, formuliert mit $\varphi$ und $\omega$ \cite{V353}:}
    A(\omega) &= -\frac{\sin (\varphi)}{\omega \text{RC}} U_0 \label{eq:A2}
  \end{align}
  \newpage
  \flushleft{Aus }\justifying Gl. \eqref{eq:A} ergibt sich eine weitere Eigenschaft des RC-Glieds. Bei niedrigen 
  Frequenzen geht $A(\omega)$ gegen $U_0$, während es für große Frequenzen
  gegen Null geht.
  Das wird im allgemeinen als Tiefpass bezeichnet und in der 
  elektrischen Schaltungstechnik häufig eingesetzt, um niedrige Frequenzen
  herauszufiltern.\\
  Ein RC-Kreis kann auch unter der Bedingung $\omega \gg \sfrac{1}{\tau}$ eine 
  zeitlich veränderliche Spannung $U(t)$ integrieren. Infolgedessen lässt sich 
  zeigen, dass $U_C(t)$ näherungseweise beschrieben werden kann durch \cite{V353}:
  \begin{align}
    U_C(t) &= \frac{1}{RC} \int_0^t{U(t') \, \symup{d}t'}\label{eq:int}
  \end{align}

% Aufgaben %%%%%%%%%%%%%%%%%%%%%%%%%%%%%%%%%%%%%%%%%%%%%%%%%%%%%%%%%%%%%%%%%%%%%%%%%%%%%%%%%%%%%%%%%%%%%%%%%%%%%%%%%%%%%%%%%%%%%%%%%%%%%%%%%%%%%%%%%%%

\section{Aufgaben}

 \begin{enumerate}
    \item[a)] \justifying Zuerst wird die Zeitkonstante $\tau$ eines RC-Glieds durch 
              eine lineare Ausgleichsrechnung bestimmt.

    \item[b)] \justifying Danach wird aus den aufgenommenen Messwerten für Spannung und 
              Frequenz mit einer nicht-linearen Ausgleichsrechnung $\tau$ berechnet 
              und ein zugehöriges $\{ A(\nu _i)/U_0,\nu _i \}$-Diagramm geplottet.
  
    \item[c)] \justifying Daraufhin wird der gleiche Prozess aus b) nochmal mit
              $\varphi$ anstelle von $A(\nu _i)/U_0$ durchgeführt.
  
    \item[d)] \justifying Zuletzt wird gezeigt, dass ein RC-Kreis unter hohen Frequenzen 
              als Integrator arbeiten kann.
  \end{enumerate}

% Versuchsaufbau %%%%%%%%%%%%%%%%%%%%%%%%%%%%%%%%%%%%%%%%%%%%%%%%%%%%%%%%%%%%%%%%%%%%%%%%%%%%%%%%%%%%%%%%%%%%%%%%%%%%%%%%%%%%%%%%%%%%%%%%%%%%%%%%%%%%%%%%%%%

\section{Versuchsaufbau}\justifying

Benötigt werden: \textit{ein analoges Zweikanal-Oszilloskop, ein Generator, ein RC-Glied, drei BNC-Kabel und ein BNC-Steckverbinder }\\
Zuerst wird an dem Generatorausgang ein BNC-Steckverbinder angebracht.\\ 
Anschließend wird der Generator mit dem RC-Glied und dem Oszilloskop verbunden. \\
Zuletzt wird das RC-Glied am zweiten Eingang des Oszilloskop angeschlossen.

% Versuchsdurchführung %%%%%%%%%%%%%%%%%%%%%%%%%%%%%%%%%%%%%%%%%%%%%%%%%%%%%%%%%%%%%%%%%%%%%%%%%%%%%%%%%%%%%%%%%%%%%%%%%%%%%%%%%%%%%%%%%%%%%%%%%%%%%%%%%%%%%%%%%%%

\newpage
\section{Versuchsdurchführung}\justifying

\begin{enumerate}
\item[a)] Für die Berechnung von $\tau$ über eine Auf- oder Entladekurve wird der 
          Generator auf eine Rechteckspannung von 0.3V geschaltet und das Oszilloskop
          auf einen geeigneten Triggerpegel eingestellt. Daraufhin wird ein Bild von der Auf- 
          oder Entladekurve der Schaltung angefertigt.\\
\item[b-c)] Für den nächsten Teil wird der Generator auf eine Sinusspannung 
          eingestellt. Dabei sollen die Spannungsverläufe von $U_C(t)$ und der 
          Generatorspannung auf dem Oszilloskop gleichzeitig angezeigt werden. 
          Dann werden die Werte für die Amplitude, der Kondensatorspannung 
          [$A(\omega$)] und dem zeitlichen Abstand der Nulldurchgänge 
          [$a$] in Abhängigkeit von der Frequenz $\nu$ über drei Zehnerpotenzen gemessen.\\
\item[d)] Um die Integrierbarkeit nachzuweisen, wird am Generator eine passende 
          Frequenz für die Integration eingestellt. Danach wird nacheinander eine 
          Rechteck-, Sinus- und Dreiecksschwingung auf das RC-Glied gegeben. 
          Dabei sollen wieder beide Spannungen auf dem Oszilloskop dargestellt 
          werden. Im Anschluss werden dann Fotos von den Oszillogrammen gemacht.
\end{enumerate}

% Auswertung %%%%%%%%%%%%%%%%%%%%%%%%%%%%%%%%%%%%%%%%%%%%%%%%%%%%%%%%%%%%%%%%%%%%%%%%%%%%%%%%%%%%%%%%%%%%%%%%%%%%%%%%%%%%%%%%%%%%%%%%%%%%%%%%%%%%%%%%%%%

\newpage
\section{Auswertung}\justifying

%Auswertung 4a ----------------------------------------------------------------------------------------------------------------------------------------
  
  Das Diagramm des Oszilloskops aus dem die Messwerte für die Tabelle \ref{tab:data4a} entnommen 
  wurden, sieht folgendermaßen aus:
  \begin{figure}[H]
    \includegraphics[width=\textwidth]{images/4a.jpg}
    \centering
    \caption{Auf- und Entladekurve des Kondensators}
    \label{fig:4ajpg}
  \end{figure}
  \flushleft{Die} Messwerte stammen dabei aus der rechten Entladungskurve.
  \begin{table}[H]
        \centering
        \caption{Messdaten von Aufg. a)}
        \input{table_4a.tex} 
        \label{tab:data4a}
  \end{table}
  \newpage
  \flushleft{Aus diesen Messwerten wurde dann mithilfe des Programms Scipy} \cite{scipy} und dem Befehl linregress() eine
  Ausgleichsgerade mit den Parametern

  \begin{align*}
    m &= \text{\input{slope.tex}}\\
    b &= \text{\input{intercept.tex}}
  \end{align*}
  berechnet, die sich folgerndermaßen darstellen lässt:
  \begin{figure}[H]
    \centering
    \includegraphics[width=\textwidth]{build/plot4a.pdf}
    \caption{Lineare Regression von Aufg. a)}
    \label{fig:4a}
  \end{figure}
  \flushleft{Bei} \justifying Betrachtung der Formel \eqref{eq:UcEnt} und der Darstellung der Wertepaare als 
  \{$ln(U_C(t_i)/U_0),t_i$\} stellt sich heraus, dass
  \begin{align}
    m = -\frac{1}{\tau} \cdot 10^{-3} \qquad \qquad
   \Rightarrow \qquad \qquad \tau_1 = \text{\input{build/mean_aRC.tex}}
  \end{align}
  Die Berechnung des Wertes $\tau$ erfolgte mit dem Programm Uncertainties \cite{uncertainties}.

% Auswertung 4b ----------------------------------------------------------------------------------------------------------------------------------------
  
  \newpage
  \flushleft{Die} \justifying für die nichtlineare Ausgleichsrechnung aus Aufgaben b) und c) benötigten Messwerte finden sich in der folgenden Tabelle:

  \begin{table}[H]
        \centering
        \caption{Messdaten von b) und c)}
        \input{table_4b.tex} 
        \label{tab:data}
  \end{table}

  \flushleft{Die} \justifying Werte aus der Tabelle für $\nu$ und $\sfrac{A(\omega)}{U_0}$ wurden in Abbildung \ref{fig:4b} graphisch
  dargestellt. Mithilfe der Formel \eqref{eq:A} und dem Befehl curve\_fit() aus Scipy \cite{scipy} wurde 
  die nichtlineare Regressionskurve für Abbildung \ref{fig:4b} berechnet:

  \begin{figure}[H]
    \includegraphics[width=\textwidth]{build/plot4b.pdf}
    \centering
    \caption{Nichtlineare Regression von Aufg. b)}
    \label{fig:4b}
  \end{figure}

  \flushleft{Aus} \justifying der curve\_fit() Funktion wird anschließend

  \begin{equation}
  \tau_2 = \text{\input{build/mean_bRC}}
  \end{equation}

  \flushleft{entnommen.} \justifying  
  Der Literaturwert für $\tau$ berechnet sich aus den Werten für die Kapazität C des Kondensators und 
  des Widerstandes R des RC-Glieds:
  \begin{equation}
  \tau_{Lit.} = \text{\input{build/L}} \label{eq:tauL}
  \end{equation}

  \flushleft{Der} \justifying relative Fehler für $\tau_1$ und $\tau_2$ ergibt dann:
  \begin{align}\label{eq:tau1_2}
    &\frac{\tau_1-\tau_{Lit.}}{\tau_{Lit.}}=\phantom{-}\text{\input{SF1.tex}} \qquad \qquad
    &\frac{\tau_2-\tau_{Lit.}}{\tau_{Lit.}}=\text{\input{SF2.tex}}
  \end{align}

  % Auswertung 4c ----------------------------------------------------------------------------------------------------------------------------------------

  \flushleft{Aus} \justifying Tabelle \ref{tab:data} werden die Frequenz $\nu$ 
  und die Zeit $\Delta T$ entnommen, um mit der Formel \eqref{eq:nu2} das 
  Bogenmaß $\varphi (\nu)$ zu bestimmen. Anschließend werden die Wertepaare $\{(\varphi(\nu_i), 
  \nu_i\}$ im folgenden Graphen \ref{fig:4c} dargestellt.

  \begin{figure}[H]
    \includegraphics[width=\textwidth]{build/plot4c.pdf}
    \centering
    \caption{Nichtlineare Regression von Aufg. c)}
    \label{fig:4c}
  \end{figure}

  \newpage
  \flushleft{Mit} \justifying $\nu$ und der Formel \eqref{eq:nu} wird die nichtlineare Regressionskurve berechnet. Dazu wird der Scipy Befehl curve\_fit() 
  \cite{scipy} verwendet. Durch die nichtlineare Regression wird die Zeitkonstante
 
  \begin{equation}
  \tau_3 = \text{\input{build/mean_cRC}}
  \end{equation}

  \flushleft{bestimmt.} \justifying Der relative Fehler für $\tau_3$ ergibt:

  \begin{equation}
  \frac{\tau_3-\tau_{Lit.}}{\tau_{Lit.}}=\text{\input{SF3.tex}}
  \end{equation}

  % Auswertung 4d ----------------------------------------------------------------------------------------------------------------------------------------

  \flushleft{Bei} \justifying den folgenden Fotos handelt es sich um die in d) zu erstellenden Oszillogramme:

  \begin{figure}[H]
  \centering
  \caption{Schwingungstypen}
  \vspace{+1em}
    \begin{subfigure}{0.48\linewidth}
      \includegraphics[width=\textwidth]{images/rechteck.jpg}
      \centering
      \caption{Rechteckschwingung für Aufg. d)}
      \label{fig:rechteck}
    \end{subfigure}
    \begin{subfigure}{0.48\linewidth}
      \includegraphics[width=\textwidth]{images/sinus.jpg}
      \centering
      \caption{Sinusschwingung für Aufg. d)}
      \label{fig:sinus}
    \end{subfigure}
    \begin{subfigure}{0.48\linewidth}
      \vspace{5mm}
      \centering
      \includegraphics[width=\textwidth]{images/dreieck.jpg}
      \caption{Dreieckschwingung für Aufg. d)} 
      \label{fig:dreieck} 
    \end{subfigure}
  
  \label{fig:schwingungen}
  \end{figure}

\newpage
\flushleft{Für} \justifying die Erstellung der Ausgleichskurve des Polarplots wurde die Formel \eqref{eq:nu}
in die Formel \eqref{eq:A2} eingesetzt und nach $\varphi$ umestellt:

\begin{equation}
  \varphi = \arccos \left(\frac{A}{U_0}\right) \label{eq:phi}
\end{equation}

  \begin{figure}[H]
    \includegraphics[width=\textwidth]{build/plot4d.pdf}
    \centering
    \caption{Polarplot von Aufg. d)}
    \label{fig:4d}
  \end{figure}
  
% Diskussion %%%%%%%%%%%%%%%%%%%%%%%%%%%%%%%%%%%%%%%%%%%%%%%%%%%%%%%%%%%%%%%%%%%%%%%%%%%%%%%%%%%%%%%%%%%%%%%%%%%%%%%%%%%%%%%%%%%%%%%%%%%%%%%%%%%%%%%%%%%

\section{Diskussion}\justifying

% Diskussion 4a ----------------------------------------------------------------------------------------------------------------------------------------

Das Bild \ref{fig:4ajpg} zeigt den erwarteten Verlauf einer Auf- und Entladekurve
eines Kondensators nach Gl. \eqref{eq:UcEnt} und \eqref{eq:UcAuf}, wobei der Kondensator bei $t=0$ fast voll aufgeladen ist.
Die Ausgleichsgerade gibt dabei den vermuteten Entladeverlauf nach Gl. \eqref{eq:UcEnt} wieder 
ohne große Abweichungen von den Messdaten.

% Diskussion 4b ----------------------------------------------------------------------------------------------------------------------------------------

\flushleft{Die} \justifying gefittete Kurve \ref{fig:4b} erfüllt nicht nach Formel \eqref{eq:A} unsere Erwartung eines Tiefpasses. 
Die Messwerte aus Tabelle \ref{tab:data} stellen allerdings die eigentliche Kurve
dar. Der Grund dafür ist, dass beim Aufschreiben der Messwerte nicht beachtet wurde,
dass der 5x-Multiplikatorknopf für Eingang 1 gedrückt war. Dies ist auch auf dem Bild \ref{fig:4ajpg}
zu sehen.

\newpage
\flushleft{Wenn} \justifying $A(\omega)$ um den Faktor $\sfrac{1}{5}$ korrigiert wird, sieht die Kurve 
folgendermaßen aus:

\begin{figure}[H]
    \includegraphics[width=\textwidth]{plot4btrue.pdf}
    \centering
    \caption{Korrigierte nichtlineare Regression von Aufg. b)}
    \label{fig:4btrue}
\end{figure}


\flushleft{Dies} \justifying zeigt die Eigenschaft des RC-Kreises als Tiefpass.
Denn bei hohen Frequenzen geht $\sfrac{A(\omega)}{U_0}$ gegen Null und bei kleinen
Frequenzen ist $\sfrac{A(\omega)}{U_0}$ ungefähr 1.
Wird der Wert $\tau_1$ \eqref{eq:tau1_2} mit dem Literaturwert $\tau_{Lit.}$
\eqref{eq:tauL} verglichen, wird festgestellt, dass der relative Fehler für $\tau_1$ 
44,2\% beträgt. Dies ist ein signifikanter Fehler, der sich mit einem  
systematischen Fehler erklären ließe. Dieser spiegelt sich unter anderem 
in der Konstanten $b$ der Ausgleichsgeraden wieder, da diese einen Wert $\ne 0$ 
annimmt. 

% Diskussion 4c ----------------------------------------------------------------------------------------------------------------------------------------

\flushleft{Der} \justifying Graph \ref{fig:4c} stellt nicht die wie erwartete Werteentwicklung nach Formel
\eqref{eq:nu} dar. Auffällig ist die Größenordung der y-Achse, da die Phase zu
springen scheint. Daraus lässt sich schlussfolgern, dass bei den ersten 14 Messwerten
für die Phase aus Tabelle \ref{tab:data} eine falsche Größenordnung vorliegt.
Erwartet wurden Werte, welche um drei Zehnerpotenzen kleiner sein sollten. 
Würde die Größenordnung korrigiert werden, sähe der neue Graph aus wie folgt:

  \begin{figure}[H]
    \includegraphics[width=\textwidth]{build/plot4ctrue.pdf}
    \centering
    \caption{Korrigierte nichtlineare Regression von Aufg. c)}
    \label{fig:4ctrue}
  \end{figure}

\flushleft{Der} \justifying korrigierte Graph beinhaltet eine arcctan-Kurve, welche aus der Theorie 
erwartet wurde. Die Streuung der einzelnen Werte lässt sich zum Teil von dem 
systematischen Fehler und zum Teil von der Messungenauigkeit des Oszilloskop erklären.
Die Ungenauigkeit hebt sich besonders bei den letzten vier Messwerten hervor. 
Eine mögliche Erklärung der großen Abweichung der letzten Messwerte ist die hohe 
Streurate der Elektronen bei erhöhten Frequenzen. Diese Ungenauigkeit führte zu einer 
konsekutiven Halbierung der Phase.

% Diskussion 4d ----------------------------------------------------------------------------------------------------------------------------------------

\flushleft{Anhand} \justifying der Gl. \eqref{eq:int} und der Bilder aus Abbildung\ref{fig:schwingungen} 
lässt sich die Integrierbarkeit darstellen.
Für den Polarplot \ref{fig:4d} setzt sich der Fehler aus Aufgabe b) und c) weiter fort.
Mithilfe der Formel \eqref{eq:phi} wird eine Sinkurve zu einer Kosinuskurve integriert.
Gleiches gilt für die Dreiecksspannung, welche zu einer parabelförmigen
Kurve für unsere Spannung $U_C(t)$ wird.
Werden die korrigierten Werte aus c) auf den Polarplot \ref{fig:4d} angewandt, ergibt sich 
folgender Polarplot:

  \begin{figure}[H]
    \includegraphics[width=\textwidth]{build/plot4dtrue.pdf}
    \centering
    \caption{Polarplot von Aufg. d)}
    \label{fig:4dtrue}
  \end{figure}

\newpage
\nocite{V353}
\nocite{scipy}
\nocite{uncertainties}
\printbibliography
\end{document}