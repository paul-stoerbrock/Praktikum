\documentclass[
  captions=tableheading,
  bibliography=totoc, 
  titepage=firstiscover,
]{scrartcl}

\usepackage{blindtext} %neuer input

\usepackage{longtable} % Tabellen über mehrere Seiten

\usepackage[utf8]{inputenc} %neuer input

\usepackage{scrhack}

\usepackage[aux]{rerunfilecheck} %Warnung falls nochmal kompiliert werden muss

\usepackage{fontspec} %Fonteinstellungen

\recalctypearea{}

\usepackage[main=ngerman]{babel} %deutsche Spracheinstellung

\usepackage{ragged2e} %neuer input

\usepackage{amsmath, nccmath}

\usepackage{amssymb} %viele mathe Symbole

\usepackage{mathtools} %Erweiterungen für amsmath

\usepackage{MnSymbol}


\DeclarePairedDelimiter{\abs}{\lvert}{\rvert}
\DeclarePairedDelimiter{\norm}{\lVert}{\rVert}

\DeclarePairedDelimiter{\bra}{\langle}{\rvert}
\DeclarePairedDelimiter{\ket}{\lvert}{\rangle}

\DeclarePairedDelimiterX{\braket}[2]{\langle}{\rangle}{
#1 \delimsize| #2
}

\usepackage{expl3}
\usepackage{xparse}
\NewDocumentCommand \dif {m}
{
\mathinner{\symup{d} #1}
}

\NewDocumentCommand \del {mm}
{
    \mathinner{\frac{\partial #1}{\partial #2}}
}
\NewDocumentCommand \deln {mmm}
{
    \mathinner{\frac{\partial^#3 #1}{\partial #2 ^#3}}
}
\ExplSyntaxOff

\usepackage[
  math-style=ISO,
  bold-style=ISO,
  sans-style=italic,
  nabla=upright,
  partial=upright,
  warnings-off={
    mathtools-colon,
    mathtools-overbracket,
  },
]{unicode-math}

\setmathfont{Latin Modern Math}
\setmathfont{XITS Math}[range={scr, bfscr}]
\setmathfont{XITS Math}[range={cal, bfcal}, StylisticSet=1]

\usepackage[
version=4,
math-greek=default,
text-greek=default,
]{mhchem}

\usepackage[
  locale=DE,
  separate-uncertainty=true,
  per-mode=reciprocal,
  output-decimal-marker={,},
]{siunitx}

\usepackage[autostyle]{csquotes} %richtige Anführungszeichen

\usepackage{xfrac}

\usepackage{float}

\floatplacement{figure}{htbp}

\floatplacement{table}{htbp}

\usepackage[ %floats innerhalb einer section halten
  section,   %floats innerhalb er section halten
  below,     %unterhalb der Section aber auf der selben Seite ist ok
]{placeins}

\usepackage[
  labelfont=bf,
  font=small,
  width=0.9\textwidth,
]{caption}

\usepackage{subcaption} %subfigure, subtable, subref

\usepackage{graphicx}

\usepackage{ulem}

\usepackage{color}

\usepackage{grffile}

\usepackage{booktabs}

\sisetup{separate-uncertainty=true}

\usepackage{microtype} %Verbesserungen am Schriftbild

\usepackage[
backend=biber,
]{biblatex}

\addbibresource{../lit.bib}

\usepackage[ %Hyperlinks im Dokument
  german,
  unicode,
  pdfusetitle,
  pdfcreator={},
  pdfproducer={},
]{hyperref}

\usepackage{bookmark}

\usepackage[shortcuts]{extdash}


\usepackage{physics}
\allowdisplaybreaks

\begin{document}
    \title{Physik IV Übungsblatt 6}
    \author{  
    Tobias Rücker\\
    \texorpdfstring{\href{mailto:tobias.ruecker@tu-dortmund.de}{tobias.ruecker@tu-dortmund.de}
    \and}{,} 
    Paul Störbrock\\
    \texorpdfstring{\href{mailto:paul.stoerbrock@tu-dortmund.de}{paul.stoerbrock@tu-dortmund.de}}{}
    }
\maketitle
\center{\Large Abgabegruppe: \textbf{4H}}
\thispagestyle{empty}

\newpage
\tableofcontents
\thispagestyle{empty}
\newpage

\setcounter{page}{1}

\section{Aufgabe 1}

    \begin{figure}[H]
        \centering
        \includegraphics[width=\textwidth]{images/Aufgabe1.jpg}
        \label{fig:1}
    \end{figure}

    \subsection{a)}

    \flushleft{Observable\;}\justifying = Hermitisch:
    \begin{align*}
    \hat{A} &= \hat{A}^{\dagger} = \left(A^T\right)^*\\
    \hat{B} &= \hat{B}^{\dagger} = \left(A^T\right)^*\\
    \hat{A} &=
    \begin{pmatrix}
        1 & 0 & 0\\
        0 & 1 & 0\\
        0 & 0 & -2
    \end{pmatrix},\;
    \hat{B} =
    \begin{pmatrix}
        0 & 0 & 0\\
        0 & 0 & -\text{i}\\
        0 & \text{i} & 0
    \end{pmatrix}\\
    \hat{A}^{\dagger} &= 
    \begin{pmatrix}
        1 & 0 & 0\\
        0 & 1 & 0\\
        0 & 0 & -2
    \end{pmatrix}^T =
    \begin{pmatrix}
        1 & 0 & 0\\
        0 & 1 & 0\\
        0 & 0 & -2
    \end{pmatrix}^* = A \Rightarrow
    \text{Hermitisch}
    \intertext{\flushleft{Operator\;}\justifying $A$ ist eine Observable.
    }
    \hat{B}^{\dagger} &= 
    \begin{pmatrix}
        0 & 0 & 0\\
        0 & 0 & \text{i}\\
        0 & -\text{i} & 0
    \end{pmatrix}^T =
    \begin{pmatrix}
        0 & 0 & 0\\
        0 & 0 & -\text{i}\\
        0 & \text{i} & 0
    \end{pmatrix}^* = B \Rightarrow
    \text{Hermitisch}
    \intertext{\flushleft{Operator\;}\justifying $B$ ist eine Observable.
    }
    \intertext{\flushleft{Die\;}\justifying möglichen Messwerte sind die EW der Observablen:
    }
    \intertext{\flushleft{EW}\justifying $(\hat{A})$:
    }
    p(\lambda) &= det\vert\hat{A}-\lambda E \vert\\
    &= det\left[
    \begin{pmatrix}
        1 & 0 & 0\\
        0 & 1 & 0\\
        0 & 0 & -2
    \end{pmatrix}
    -
    \begin{pmatrix}
        \lambda & 0 & 0\\
        0 & \lambda & 0\\
        0 & 0 & \lambda
    \end{pmatrix}
     \right]\\
    &= det
    \begin{vmatrix}
        1-\lambda & 0 & 0\\
        0 & 1-\lambda & 0\\
        0 & 0 & -2-\lambda
    \end{vmatrix}\\
    &= (1-\lambda)(1-\lambda)(-2-\lambda)\\
    &\Rightarrow \lambda_{1,2} = 1 \qquad
    \lambda_3 =-2
    \intertext{\flushleft{EW}\justifying $(\hat{B})$:
    }
    p(\lambda) &= det\vert\hat{B}-\lambda E \vert\\
    &=det\left[
    \begin{pmatrix}
        0 & 0 & 0\\
        0 & 0 & -\text{i}\\
        0 & \text{i} & 0
    \end{pmatrix}
    -
    \begin{pmatrix}
        \lambda & 0 & 0\\
        0 & \lambda & 0\\
        0 & 0 & \lambda
    \end{pmatrix}
    \right]\\
    &= 
    \begin{vmatrix}
        -\lambda & 0 & 0\\
        0 & -\lambda & -\text{i}\\
        0 & \text{i} & -\lambda
    \end{vmatrix}\\
    &=(-\lambda)(-\lambda)(-\lambda)-(\text{i})(-\text{i})(-\lambda)\\
    &=-\lambda^3+\lambda\\
    &= \lambda(1-\lambda^2)\\
    &\Rightarrow \lambda_1 = 0 \qquad
    \lambda_2 = 1 \qquad \lambda_3 =-1
    \end{align*}

    \subsection{b)}

    \flushleft{Können\;}\justifying $\hat{A}$ und $\hat{B}$ gleichzeitig gemessen werden? $\Leftrightarrow$ Kommutieren $\hat{A}$ und $\hat{B}$?
    \begin{align*}
        \left[ \hat{A},\hat{B} \right] &= \hat{A}\hat{B}-\hat{B}\hat{A}\\
    \begin{pmatrix}
        1 & 0 & 0\\
        0 & 1 & 0\\
        0 & 0 & -2
    \end{pmatrix}
    \begin{pmatrix}
        0 & 0 & 0\\
        0 & 0 & -\text{i}\\
        0 & \text{i} & 0
    \end{pmatrix}
    &=
    \begin{pmatrix}
        0 & 0 & 0\\
        0 & 0 & -\text{i}\\
        0 & -2\text{i} & 0
    \end{pmatrix}\\
    \begin{pmatrix}
        0 & 0 & 0\\
        0 & 0 & -\text{i}\\
        0 & \text{i} & 0
    \end{pmatrix}
    \begin{pmatrix}
        1 & 0 & 0\\
        0 & 1 & 0\\
        0 & 0 & -2
    \end{pmatrix}
    &=
    \begin{pmatrix}
        0 & 0 & 0\\
        0 & 0 & 2\text{i}\\
        0 & \text{i} & 0
    \end{pmatrix}\\
    \begin{pmatrix}
        0 & 0 & 0\\
        0 & 0 & -\text{i}\\
        0 & -2\text{i} & 0
    \end{pmatrix}
    -
    \begin{pmatrix}
        0 & 0 & 0\\
        0 & 0 & 2\text{i}\\
        0 & \text{i} & 0
    \end{pmatrix}
    &=
    \begin{pmatrix}
        0 & 0 & 0\\
        0 & 0 & -3\text{i}\\
        0 & -3\text{i} & 0
    \end{pmatrix} \neq 0
    \intertext{\flushleft{Der\;}\justifying Kommutator ist $\neq$ 0, also können $\hat{A}$ und $\hat{B}$ nicht gleichzeitig gemessen werden.
    }
    \end{align*}

    \subsection{c)}

    \flushleft{Zuerst\;}\justifying werden die EV der Observablen bestimmt:
    \begin{align*}
    \intertext{\flushleft{EV}\justifying $(\hat{A})$:
    }
    &\left( \hat{A}-\lambda E \right) \vec{v} = \vec{0}\;\text{für}\;\lambda = 1\\
    &\begin{pmatrix}
        1-\lambda & 0 & 0\\
        0 & 1-\lambda & 0\\
        0 & 0 & -2-\lambda
    \end{pmatrix}
    \begin{pmatrix}
        v_1\\
        v_2\\
        v_3
    \end{pmatrix}
    =
    \begin{pmatrix}
        0\\
        0\\
        0
    \end{pmatrix}\\
    &\begin{pmatrix}
        0\\
        0\\
        -3v_3
    \end{pmatrix}
    =
    \begin{pmatrix}
        0\\
        0\\
        0
    \end{pmatrix}\\
    &\Rightarrow v_{1,2} \in \mathbb{R} \qquad v_3=0\\
    \\
    &\left( \hat{A}-\lambda E \right) \vec{v} = \vec{0}\;\text{für}\;\lambda = -2\\
    &\begin{pmatrix}
        1-\lambda & 0 & 0\\
        0 & 1-\lambda & 0\\
        0 & 0 & -2-\lambda
    \end{pmatrix}
    \begin{pmatrix}
        v_1\\
        v_2\\
        v_3
    \end{pmatrix}
    =
    \begin{pmatrix}
        0\\
        0\\
        0
    \end{pmatrix}\\
    &\begin{pmatrix}
        3v_1\\
        3v_2\\
        0
    \end{pmatrix}
    =
    \begin{pmatrix}
        0\\
        0\\
        0
    \end{pmatrix}\\
    &\Rightarrow v_{1,2}=0 \qquad v_3 \in \mathbb{R}
    \intertext{\flushleft{EV}\justifying $(\hat{B})$:
    }
    &\left( \hat{B}-\lambda E \right) \vec{v} = \vec{0}\;\text{für}\;\lambda = 0\\
    &\begin{pmatrix}
        -\lambda & 0 & 0\\
        0 & -\lambda & -\text{i}\\
        0 & \text{i} & -\lambda
    \end{pmatrix}
    \begin{pmatrix}
        v_1\\
        v_2\\
        v_3
    \end{pmatrix}
    =
    \begin{pmatrix}
        0\\
        0\\
        0
    \end{pmatrix}\\
    &\begin{pmatrix}
        0\\
        -\text{i}v_3\\
        \text{i}v_2
    \end{pmatrix}
    =
    \begin{pmatrix}
        0\\
        0\\
        0
    \end{pmatrix}\\
    &\Rightarrow v_1 \in \mathbb{R} \qquad v_{2,3}=0\\
    \\
    &\left( \hat{B}-\lambda E \right) \vec{v} = \vec{0}\;\text{für}\;\lambda = 1\\
    &\begin{pmatrix}
        -\lambda & 0 & 0\\
        0 & -\lambda & -\text{i}\\
        0 & \text{i} & -\lambda
    \end{pmatrix}
    \begin{pmatrix}
        v_1\\
        v_2\\
        v_3
    \end{pmatrix}
    =
    \begin{pmatrix}
        0\\
        0\\
        0
    \end{pmatrix}\\
    &\begin{pmatrix}
        -v_1\\
        -v_2-\text{i}v_3\\
        \text{i}v_2-v_3
    \end{pmatrix}
    =
    \begin{pmatrix}
        0\\
        0\\
        0
    \end{pmatrix}\\
    &\Rightarrow v_1=0 \qquad v_{2,3}\in \mathbb{R}\\
    \\
    &\left( \hat{B}-\lambda E \right) \vec{v} = \vec{0}\;\text{für}\;\lambda = -1\\
    &\begin{pmatrix}
        -\lambda & 0 & 0\\
        0 & -\lambda & -\text{i}\\
        0 & \text{i} & -\lambda
    \end{pmatrix}
    \begin{pmatrix}
        v_1\\
        v_2\\
        v_3
    \end{pmatrix}
    =
    \begin{pmatrix}
        0\\
        0\\
        0
    \end{pmatrix}\\
    &\begin{pmatrix}
        v_1\\
        v_2-\text{i}v_3\\
        \text{i}v_2+v_3
    \end{pmatrix}
    =
    \begin{pmatrix}
        0\\
        0\\
        0
    \end{pmatrix}\\
    &\Rightarrow v_1=0 \qquad v_{2,3}\in \mathbb{R}
    \intertext{\flushleft{Aus\;}\justifying den EV der Observablen lassen sich die Eigenzustände $\ket{\lambda_i}$ der jeweiligen 
    Messwerte $\lambda_i$ bestimmen. Dazu werden frei wählbare Komponenten der EV mit einem Vorfaktor \{a,b,c\} vor dem Eigenzustand dargestellt.
    Da die Messung von $\hat{B}$ als Ergebnis 1 ergibt, handelt es sich hier um den Eigenzustand zum EW $\lambda_2$= 1. Hier sind die Komponenten
    $v_{2,3}$ frei wählbar, es folgt also für den Eigenzustand $\ket{\lambda_{B_2}}$:
    }
    \ket{\lambda_{B_2}} &= b\ket{2}+c\ket{3}
    \end{align*}

    \subsection{d)}

    \flushleft{Die\;}\justifying möglichen Messergebnisse des Operators $\hat{A}$ sind $\lambda_{1,2}=1$ und $\lambda_3=-2$. Mit den jeweiligen
    Zuständen wird analog zum Aufgabenteil c) verfahren:
    \begin{align*}
    \ket{\lambda_{A_{1,2}}} &= a\ket{1}+b\ket{2}\\
    \ket{\lambda_{A_3}} &= c\ket{3}
    \end{align*}
    Für die Wahrscheinlichkeiten der möglichen Messergebnisse der Anschlussmessung gilt: 
    \begin{align*}
    P(\lambda_{A_1}) &= P(\lambda_{A_2}) = \abs{\braket{\lambda_{A_{1,2}}}{\lambda_{B_2}}}^2 =\braket{\lambda_{B_2}}{\lambda_{A_{1,2}}}\braket{\lambda_{A_{1,2}}}{\lambda_{B_2}}\\
    &= \left( -\frac{\text{i}}{\sqrt{2}}\ket{2}+\frac{1}{\sqrt{2}}\ket{3} \right) \left( \frac{1}{\sqrt{2}} \ket{1}+\frac{1}{\sqrt{2}}\ket{2} \right)
    \left( \frac{1}{\sqrt{2}}\ket{1}+\frac{1}{\sqrt{2}}\ket{2} \right) \left( \frac{\text{i}}{\sqrt{2}} \ket{2}+\frac{1}{\sqrt{2}}\ket{3} \right)\\
    &= \left( -\frac{\text{i}}{2} \underbrace{\braket{2}{1}}_{=0} - \frac{\text{i}}{2} \underbrace{\braket{2}{2}}_{=1} + 
    \frac{1}{2} \underbrace{\braket{3}{1}}_{=0} + \frac{1}{2} \underbrace{\braket{3}{2}}_{=0} \right)
    \left( \frac{\text{i}}{2} \underbrace{\braket{1}{2}}_{=0} + \frac{1}{2} \underbrace{\braket{1}{3}}_{=0} + 
    \frac{\text{i}}{2} \underbrace{\braket{2}{2}}_{=1} + \frac{1}{2} \underbrace{\braket{2}{3}}_{=0} \right)\\
    &= \frac{-\text{i}}{2} \cdot \frac{\text{i}}{2} = \frac{1}{4}\\
    P(\lambda_{A_3}) &= \abs{\braket{\lambda_{A_3}}{\lambda_{B_2}}}^2 =\braket{\lambda_{B_2}}{\lambda_{A_3}}\braket{\lambda_{A_3}}{\lambda_{B_2}}\\
    &= \left( -\frac{\text{i}}{\sqrt{2}}\ket{2}+\frac{1}{\sqrt{2}}\ket{3} \right) \ket{3}
    \bra{3} \left( \frac{\text{i}}{\sqrt{2}} \ket{2}+\frac{1}{\sqrt{2}}\ket{3} \right)\\
    &= \left( -\frac{\text{i}}{\sqrt{2}} \underbrace{\braket{2}{3}}_{=0} + \frac{1}{2} \underbrace{\braket{3}{3}}_{=1} + 
    \frac{\text{i}}{\sqrt{2}} \underbrace{\braket{3}{2}}_{=0} + \frac{1}{2} \underbrace{\braket{3}{3}}_{=1} \right)\\
    &= \frac{1}{\sqrt{2}} \cdot \frac{1}{\sqrt{2}} = \frac{1}{2}
    \end{align*}

    \subsection{e)}

    \subsection{f)}


\section{Aufgabe 2}

    \begin{figure}[H]
        \centering
        \includegraphics[width=\textwidth]{images/Aufgabe2abc.jpg}
        \label{fig:2}
    \end{figure}

    \begin{figure}[H]
        \centering
        \includegraphics[width=\textwidth]{images/Aufgabe2def.jpg}
        \label{fig:3}
    \end{figure}

    \subsection{a)}
    \begin{align}
    \mathrm{Z\kern-.3em\raise-0.5ex\hbox{Z}} [\hat A, \hat B \hat C] &= [\hat A , \hat B]\hat C + \hat B [\hat A , \hat C]\\
    [hat A , \hat B \hat C] &= \hat A \hat B \hat C - \hat B \hat C \hat A -\hat B \hat A \hat C + \hat B \hat A \hat C\\
    &=(\hat A \hat B - \hat B \hat A) \hat C + \hat B (\hat A \hat C -\hat C \hat A)\\
    &=\underline{\underline{[\hat A , \hat B]\hat C + \hat B [\hat A , \hat C]}}
    \end{align}
    \subsection{b)}
    \begin{align}
    \mathrm{Z\kern-.3em\raise-0.5ex\hbox{Z}} &[\hat A , [\hat B , \hat C]] + [\hat B ,[\hat C, \hat A]]+ [\hat C,[\hat A , \hat B]] =0\\
    [\hat A , [\hat B , \hat C]] &= [\hat A, \hat B \hat C - \hat C \hat B] \\
    &=[\hat A , \hat B \hat C] -[\hat A , \hat C \hat B]\\
    &= \hat A \hat B \hat C -\hat B \hat C \hat A- (\hat A \hat C \hat B-\hat C \hat B \hat A)\\
    [\hat B , [\hat C , \hat A]] &= [\hat B, \hat C \hat A-\hat A \hat C]\\
    &= \hat B \hat C \hat A - \hat C \hat A \hat B - (\hat B \hat A \hat C - \hat A \hat C \hat B)\\
    [\hat C , [\hat A , \hat B]] &= [\hat C , \hat A \hat B]- [\hat C , \hat B \hat A]\\
    &= \hat C \hat A \hat B - \hat A \hat B \hat C -(\hat C \hat B \hat A - \hat B \hat A \hat C)\\
    \Rightarrow &[\hat A , [\hat B , \hat C]] + [\hat B , [\hat C , \hat A]] +[\hat C , [\hat A , \hat B]] = 0
    \end{align}
    \subsection{c)}
    \begin{align}
    \mathrm{Z\kern-.3em\raise-0.5ex\hbox{Z}} [\hat A , \hat B ^n] &= \sum_{i=1}^n \hat B ^{i-1} [\hat A , \hat B ] \hat B ^{n-i}\\
    [\hat A , \hat B ^n] &= [\hat a , \hat B \hat B ^{n-1}]\\
    &= [\hat A , \hat B] \hat B ^{n-1} + \hat B [\hat A , \hat B ^{n-1}]\\
    &= [\hat A , \hat B] \hat B ^{n-1} + \hat B [\hat A , \hat B \hat B ^{n-2}] \\
    &= [\hat A , \hat B] \hat B ^{n-1} + \hat B ([\hat A , \hat B] \hat B ^{n-2} + \hat B [\hat A , \hat B ^ {n-2} ] )\\
    &= [\hat A , \hat B] \hat B ^{n-1} + \hat B [\hat A , \hat B] \hat B ^{n-2} + \hat B ^2 [\hat  A , \hat B ^{n-2}]\\
    &= [\hat A , \hat B] \hat B ^{n-1} + \hat B [\hat A, \hat B] \hat B ^{n-2} + \hdots + \hat B ^{n-1} [\hat A , \hat B]\\
    &=\underline{\underline{ \sum_{i=1}^n \hat B ^{i-1} [\hat A , \hat B] \hat B ^{n-i}}}
    \end{align}
    \subsection{d)}
    \begin{align}
    \mathrm{Z\kern-.3em\raise-0.5ex\hbox{Z}} [\hat A , f8\hat B] &= f'(\hat B) [\hat A , \hat B]\\
    \intertext{
        Potenzriehenansatz für die Funktion f:
    }
    f(\hat B) &= \sum_{n=0}^{\infty} a_n \hat B ^n\\
    f'(\hat B) &= \sum_{n=1}^{\infty} a_n \hat B ^n \\
    [\hat A , \sum_{n=0}^{\infty} a_n \hat B ^n ] &= \sum_{n=0}^{\infty} a_n [\hat A , \hat B ^n]
    \intertext{
        0-tes Summenglied verschwindet und anwenden von c)
    }
    &=\sum_{n=1}^{\infty} a_n \sum_{i=1}^n \hat B ^{i-1} [\hat A , \hat B] \hat B ^{n-i}\\
    \intertext{
        Dadurch, dass der Kommutator $[[\hat A , \hat B], \hat B]=0$ ist, können die beiden Terme vertauscht werden.
    }
    &=\sum_{n=1}^{\infty} a_n \sum_{i=1}^n \hat B ^{i-1} \hat B ^{n-i} [\hat A , \hat B]\\
    &=\sum_{n=1}^{\infty} a_n \sum_{i=1}^n \hat B^{n-1} [\hat A , \hat B]\\
    &=\sum_{n=1}^{\infty} a_n n \hat B ^{n-1} [\hat A , \hat B]\\
    &= f'(\hat B) [\hat A , \hat B]
    \end{align}
    
    \subsection{e)}
    \begin{align}
    [\hat x , f(\hat p)] &= f' (\hat p ) [\hat x , \hat p] = i \hbar f' (\hat p)
    \end{align}
    \subsection{f)}
    \begin{align}
    \hat A \hat B &= (\hat A \hat B)\\
    &= \hat B ^* \hat A ^*\\
    \Rightarrow & \hat A = \hat B ^* \quad \text{und} \quad \hat B = \hat A ^* 
    \end{align}


\section{Aufgabe 3}

    \begin{figure}[H]
        \centering
        \includegraphics[width=\textwidth]{images/Aufgabe3.jpg}
        \label{fig:4}
    \end{figure}

    \subsection{a)}

    \begin{align*}
        \ket{\to} &= \frac{1}{\sqrt{2}}\left(\ket{\uparrow}+\ket{\downarrow}\right), \qquad \ket{\leftarrow} = \frac{1}{\sqrt{2}}\left(\ket{\uparrow}-\ket{\downarrow}\right)\\
        \Rightarrow\bra{\to} &= \frac{1}{\sqrt{2}}\left(\bra{\uparrow}+\bra{\downarrow}\right), \qquad \bra{\leftarrow} = \frac{1}{\sqrt{2}}\left(\bra{\uparrow}-\bra{\downarrow}\right)
        \intertext{\flushleft{Normierung:\;}\justifying
        }
        \braket{\to} &= \left( \frac{1}{\sqrt{2}}\left(\bra{\uparrow}+\bra{\downarrow}\right) \right) \left(  \frac{1}{\sqrt{2}}\left(\ket{\uparrow}+\ket{\downarrow}\right) \right)\\
        &= \frac{1}{2} \underbrace{\braket{\uparrow}}_{=1} + \frac{1}{2} \underbrace{\braket{\uparrow}{\downarrow}}_{=0}
        + \frac{1}{2} \underbrace{\braket{\downarrow}{\uparrow}}_{=0} + \frac{1}{2} \underbrace{\braket{\downarrow}{\downarrow}}_{=1}\\
        &= \frac{1}{2}+\frac{1}{2} = 1\\
        \\
        \braket{\leftarrow} &= \left( \frac{1}{\sqrt{2}}\left(\bra{\uparrow}-\bra{\downarrow}\right) \right) \left(  \frac{1}{\sqrt{2}}\left(\ket{\uparrow}-\ket{\downarrow}\right) \right)\\
        &= \frac{1}{2} \underbrace{\braket{\uparrow}}_{=1} - \frac{1}{2} \underbrace{\braket{\uparrow}{\downarrow}}_{=0}
        - \frac{1}{2} \underbrace{\braket{\downarrow}{\uparrow}}_{=0} + \frac{1}{2} \underbrace{\braket{\downarrow}{\downarrow}}_{=1}\\
        &= \frac{1}{2}+\frac{1}{2} = 1
        \intertext{\flushleft{Orthogonalität:\;}\justifying
        }
        \braket{\leftarrow}{\to} &= \left( \frac{1}{\sqrt{2}}\left(\bra{\uparrow}-\bra{\downarrow}\right) \right) \left(  \frac{1}{\sqrt{2}}\left(\ket{\uparrow}+\ket{\downarrow}\right) \right)\\
        &= \frac{1}{2} \underbrace{\braket{\uparrow}}_{=1} + \frac{1}{2} \underbrace{\braket{\uparrow}{\downarrow}}_{=0}
        - \frac{1}{2} \underbrace{\braket{\downarrow}{\uparrow}}_{=0} - \frac{1}{2} \underbrace{\braket{\downarrow}{\downarrow}}_{=1}\\
        &= \frac{1}{2}-\frac{1}{2} = 0
    \end{align*}

    \subsection{b)}

    \begin{align*}
        \ket{\otimes} &= \frac{1}{\sqrt{2}}\left(\ket{\uparrow}+\text{i}\ket{\downarrow}\right), \qquad \ket{\odot} = \frac{1}{\sqrt{2}}\left(\ket{\uparrow}-\text{i}\ket{\downarrow}\right)\\
        \Rightarrow\bra{\otimes} &= \frac{1}{\sqrt{2}}\left(\bra{\uparrow}-\bra{\downarrow}\text{i}\right), \qquad \bra{\odot} = \frac{1}{\sqrt{2}}\left(\bra{\uparrow}+\bra{\downarrow}\text{i}\right)
        \intertext{\flushleft{Normierung:\;}\justifying
        }
        \braket{\otimes} &= \left( \frac{1}{\sqrt{2}}\left(\bra{\uparrow}-\bra{\downarrow}\text{i}\right) \right) \left(  \frac{1}{\sqrt{2}}\left(\ket{\uparrow}+\text{i}\ket{\downarrow}\right) \right)\\
        &= \frac{1}{2} \underbrace{\braket{\uparrow}}_{=1} + \frac{\text{i}}{2} \underbrace{\braket{\uparrow}{\downarrow}}_{=0}
        - \frac{\text{i}}{2} \underbrace{\braket{\downarrow}{\uparrow}}_{=0} - \frac{\text{i}^2}{2} \underbrace{\braket{\downarrow}{\downarrow}}_{=1}\\
        &= \frac{1}{2}-\frac{\text{i}^2}{2} = 1\\
        \braket{\odot} &= \left( \frac{1}{\sqrt{2}}\left(\bra{\uparrow}+\bra{\downarrow}\text{i}\right) \right) \left(  \frac{1}{\sqrt{2}}\left(\ket{\uparrow}-\text{i}\ket{\downarrow}\right) \right)\\
        &= \frac{1}{2} \underbrace{\braket{\uparrow}}_{=1} - \frac{\text{i}}{2} \underbrace{\braket{\uparrow}{\downarrow}}_{=0}
        + \frac{\text{i}}{2} \underbrace{\braket{\downarrow}{\uparrow}}_{=0} - \frac{\text{i}^2}{2} \underbrace{\braket{\downarrow}{\downarrow}}_{=1}\\
        &= \frac{1}{2}-\frac{\text{i}^2}{2} = 1
        \intertext{\flushleft{Orthogonalität:\;}\justifying
        }
        \braket{\otimes}{\odot} &= \left( \frac{1}{\sqrt{2}}\left(\bra{\uparrow}-\bra{\downarrow}\text{i}\right) \right) \left(  \frac{1}{\sqrt{2}}\left(\ket{\uparrow}-\text{i}\ket{\downarrow}\right) \right)\\
        &= \frac{1}{2} \underbrace{\braket{\uparrow}}_{=1} - \frac{\text{i}}{2} \underbrace{\braket{\uparrow}{\downarrow}}_{=0}
        - \frac{\text{i}}{2} \underbrace{\braket{\downarrow}{\uparrow}}_{=0} + \frac{\text{i}^2}{2} \underbrace{\braket{\downarrow}{\downarrow}}_{=1}\\
        &= \frac{1}{2}+\frac{\text{i}^2}{2} = 0
    \end{align*}

    \subsection{c)}

    \begin{align*}
    \braket{\odot}{\uparrow}\braket{\uparrow}{\odot} &= 
    \left( \frac{1}{\sqrt{2}} \underbrace{\braket{\uparrow}{\uparrow}}_{=1} + \frac{\text{i}}{\sqrt{2}} \underbrace{\braket{\downarrow}{\uparrow}}_{=0} \right)
    \left( \frac{1}{\sqrt{2}} \underbrace{\braket{\uparrow}{\uparrow}}_{=1} - \frac{\text{i}}{\sqrt{2}} \underbrace{\braket{\uparrow}{\downarrow}}_{=0} \right)\\
    &= \frac{1}{\sqrt{2}} \cdot \frac{1}{\sqrt{2}} = \frac{1}{2}\\
    \braket{\odot}{\downarrow}\braket{\downarrow}{\odot} &= 
    \left( \frac{1}{\sqrt{2}} \underbrace{\braket{\uparrow}{\downarrow}}_{=0} + \frac{\text{i}}{\sqrt{2}} \underbrace{\braket{\downarrow}{\downarrow}}_{=1} \right)
    \left( \frac{1}{\sqrt{2}} \underbrace{\braket{\downarrow}{\uparrow}}_{=0} - \frac{\text{i}}{\sqrt{2}} \underbrace{\braket{\downarrow}{\downarrow}}_{=1} \right)\\
    &= \frac{i}{\sqrt{2}} \cdot \frac{(-i)}{\sqrt{2}} = \frac{1}{2}\\
    \braket{\otimes}{\uparrow}\braket{\uparrow}{\otimes} &= 
    \left( \frac{1}{\sqrt{2}} \underbrace{\braket{\uparrow}{\uparrow}}_{=1} - \frac{\text{i}}{\sqrt{2}} \underbrace{\braket{\downarrow}{\uparrow}}_{=0} \right)
    \left( \frac{1}{\sqrt{2}} \underbrace{\braket{\uparrow}{\uparrow}}_{=1} + \frac{\text{i}}{\sqrt{2}} \underbrace{\braket{\uparrow}{\downarrow}}_{=0} \right)\\
    &= \frac{1}{\sqrt{2}} \cdot \frac{1}{\sqrt{2}} = \frac{1}{2}\\
    \braket{\otimes}{\downarrow}\braket{\downarrow}{\otimes} &= 
    \left( \frac{1}{\sqrt{2}} \underbrace{\braket{\uparrow}{\downarrow}}_{=0} - \frac{\text{i}}{\sqrt{2}} \underbrace{\braket{\downarrow}{\downarrow}}_{=1} \right)
    \left( \frac{1}{\sqrt{2}} \underbrace{\braket{\downarrow}{\uparrow}}_{=0} + \frac{\text{i}}{\sqrt{2}} \underbrace{\braket{\downarrow}{\downarrow}}_{=1} \right)\\
    &= \frac{-i}{\sqrt{2}} \cdot \frac{i}{\sqrt{2}} = \frac{1}{2}\\
    \end{align*}

    \flushleft{Die\;}\justifying $[...]$-Klammern sind hier keine Kommutatorklammern!
    \begin{align*}
        \braket{\odot}{\to}\braket{\to}{\odot} &=
    \left[ \left( \frac{1}{\sqrt{2}}\left(\bra{\uparrow}+\bra{\downarrow}\text{i}\right) \right) 
    \left( \frac{1}{\sqrt{2}}\left(\ket{\uparrow}+\ket{\downarrow}\right) \right) \right] 
    \left[ \left( \frac{1}{\sqrt{2}}\left(\bra{\uparrow}+\bra{\downarrow}\right) \right) 
    \left( \frac{1}{\sqrt{2}}\left(\ket{\uparrow}-\text{i}\ket{\downarrow}\right) \right) \right]\\
    &= \left( \frac{1}{2} \underbrace{\braket{\uparrow}{\uparrow}}_{=1} 
    + \frac{1}{2} \underbrace{\braket{\uparrow}{\downarrow}}_{=0}
    + \frac{\text{i}}{2} \underbrace{\braket{\downarrow}{\uparrow}}_{=0} 
    + \frac{\text{i}}{2} \underbrace{\braket{\downarrow}{\downarrow}}_{=1} \right)
    \left( \frac{1}{2} \underbrace{\braket{\uparrow}{\uparrow}}_{=1} 
    - \frac{\text{i}}{2} \underbrace{\braket{\uparrow}{\downarrow}}_{=0}
    + \frac{1}{2} \underbrace{\braket{\downarrow}{\uparrow}}_{=0} 
    - \frac{\text{i}}{2} \underbrace{\braket{\downarrow}{\downarrow}}_{=1} \right)\\
    &= \left( \frac{1}{2} + \frac{\text{i}}{2} \right) \left( \frac{1}{2} - \frac{\text{i}}{2} \right)\\
    &= \frac{1}{4} - \frac{\text{i}}{4} + \frac{1}{4} + \frac{\text{i}}{4} = \frac{1}{2}\\
        \braket{\odot}{\leftarrow}\braket{\leftarrow}{\odot} &=
    \left[ \left( \frac{1}{\sqrt{2}}\left(\bra{\uparrow}+\bra{\downarrow}\text{i}\right) \right) 
    \left( \frac{1}{\sqrt{2}}\left(\ket{\uparrow}-\ket{\downarrow}\right) \right) \right] 
    \left[ \left( \frac{1}{\sqrt{2}}\left(\bra{\uparrow}-\bra{\downarrow}\right) \right) 
    \left( \frac{1}{\sqrt{2}}\left(\ket{\uparrow}-\text{i}\ket{\downarrow}\right) \right) \right]\\
    &= \left( \frac{1}{2} \underbrace{\braket{\uparrow}{\uparrow}}_{=1} 
    - \frac{1}{2} \underbrace{\braket{\uparrow}{\downarrow}}_{=0}
    + \frac{\text{i}}{2} \underbrace{\braket{\downarrow}{\uparrow}}_{=0} 
    - \frac{\text{i}}{2} \underbrace{\braket{\downarrow}{\downarrow}}_{=1} \right)
    \left( \frac{1}{2} \underbrace{\braket{\uparrow}{\uparrow}}_{=1} 
    - \frac{\text{i}}{2} \underbrace{\braket{\uparrow}{\downarrow}}_{=0}
    - \frac{1}{2} \underbrace{\braket{\downarrow}{\uparrow}}_{=0} 
    + \frac{\text{i}}{2} \underbrace{\braket{\downarrow}{\downarrow}}_{=1} \right)\\
    &= \left( \frac{1}{2} - \frac{\text{i}}{2} \right) \left( \frac{1}{2} + \frac{\text{i}}{2} \right)\\
    &= \frac{1}{4} + \frac{\text{i}}{4} - \frac{1}{4} + \frac{\text{i}}{4} = \frac{1}{2}\\
        \braket{\otimes}{\to}\braket{\to}{\otimes} &=
    \left[ \left( \frac{1}{\sqrt{2}}\left(\bra{\uparrow}-\bra{\downarrow}\text{i}\right) \right) 
    \left( \frac{1}{\sqrt{2}}\left(\ket{\uparrow}+\ket{\downarrow}\right) \right) \right] 
    \left[ \left( \frac{1}{\sqrt{2}}\left(\bra{\uparrow}+\bra{\downarrow}\right) \right) 
    \left( \frac{1}{\sqrt{2}}\left(\ket{\uparrow}+\text{i}\ket{\downarrow}\right) \right) \right]\\
    &= \left( \frac{1}{2} \underbrace{\braket{\uparrow}{\uparrow}}_{=1} 
    + \frac{1}{2} \underbrace{\braket{\uparrow}{\downarrow}}_{=0}
    - \frac{\text{i}}{2} \underbrace{\braket{\downarrow}{\uparrow}}_{=0} 
    - \frac{\text{i}}{2} \underbrace{\braket{\downarrow}{\downarrow}}_{=1} \right)
    \left( \frac{1}{2} \underbrace{\braket{\uparrow}{\uparrow}}_{=1} 
    + \frac{\text{i}}{2} \underbrace{\braket{\uparrow}{\downarrow}}_{=0}
    + \frac{1}{2} \underbrace{\braket{\downarrow}{\uparrow}}_{=0} 
    + \frac{\text{i}}{2} \underbrace{\braket{\downarrow}{\downarrow}}_{=1} \right)\\
    &= \left( \frac{1}{2} - \frac{\text{i}}{2} \right) \left( \frac{1}{2} + \frac{\text{i}}{2} \right)\\
    &= \frac{1}{4} + \frac{\text{i}}{4} - \frac{1}{4} + \frac{\text{i}}{4} = \frac{1}{2}\\
        \braket{\otimes}{\leftarrow}\braket{\leftarrow}{\otimes} &=
    \left[ \left( \frac{1}{\sqrt{2}}\left(\bra{\uparrow}-\bra{\downarrow}\text{i}\right) \right) 
    \left( \frac{1}{\sqrt{2}}\left(\ket{\uparrow}-\ket{\downarrow}\right) \right) \right] 
    \left[ \left( \frac{1}{\sqrt{2}}\left(\bra{\uparrow}-\bra{\downarrow}\right) \right) 
    \left( \frac{1}{\sqrt{2}}\left(\ket{\uparrow}+\text{i}\ket{\downarrow}\right) \right) \right]\\
    &= \left( \frac{1}{2} \underbrace{\braket{\uparrow}{\uparrow}}_{=1} 
    - \frac{1}{2} \underbrace{\braket{\uparrow}{\downarrow}}_{=0}
    - \frac{\text{i}}{2} \underbrace{\braket{\downarrow}{\uparrow}}_{=0} 
    + \frac{\text{i}}{2} \underbrace{\braket{\downarrow}{\downarrow}}_{=1} \right)
    \left( \frac{1}{2} \underbrace{\braket{\uparrow}{\uparrow}}_{=1} 
    + \frac{\text{i}}{2} \underbrace{\braket{\uparrow}{\downarrow}}_{=0}
    - \frac{1}{2} \underbrace{\braket{\downarrow}{\uparrow}}_{=0} 
    - \frac{\text{i}}{2} \underbrace{\braket{\downarrow}{\downarrow}}_{=1} \right)\\
    &= \left( \frac{1}{2} + \frac{\text{i}}{2} \right) \left( \frac{1}{2} - \frac{\text{i}}{2} \right)\\
    &= \frac{1}{4} - \frac{\text{i}}{4} + \frac{1}{4} + \frac{\text{i}}{4} = \frac{1}{2}
    \end{align*}

    \subsection{d)}


\section{Aufgabe 4}

    \begin{figure}[H]
        \centering
        \includegraphics[width=\textwidth]{images/Aufgabe4.jpg}
        \label{fig:5}
    \end{figure}

    \subsection{a)}

    \subsection{b)}



\end{document}