\documentclass[
  captions=tableheading,
  bibliography=totoc, 
  titepage=firstiscover,
]{scrartcl}

\usepackage{blindtext} %neuer input

\usepackage{longtable} % Tabellen über mehrere Seiten

\usepackage[utf8]{inputenc} %neuer input

\usepackage{scrhack}

\usepackage[aux]{rerunfilecheck} %Warnung falls nochmal kompiliert werden muss

\usepackage{fontspec} %Fonteinstellungen

\recalctypearea{}

\usepackage[main=ngerman]{babel} %deutsche Spracheinstellung

\usepackage{ragged2e} %neuer input

\usepackage{amsmath, nccmath}

\usepackage{amssymb} %viele mathe Symbole

\usepackage{mathtools} %Erweiterungen für amsmath

\usepackage{MnSymbol}


\DeclarePairedDelimiter{\abs}{\lvert}{\rvert}
\DeclarePairedDelimiter{\norm}{\lVert}{\rVert}

\DeclarePairedDelimiter{\bra}{\langle}{\rvert}
\DeclarePairedDelimiter{\ket}{\lvert}{\rangle}

\DeclarePairedDelimiterX{\braket}[2]{\langle}{\rangle}{
#1 \delimsize| #2
}

\usepackage{expl3}
\usepackage{xparse}
\NewDocumentCommand \dif {m}
{
\mathinner{\symup{d} #1}
}

\NewDocumentCommand \del {mm}
{
    \mathinner{\frac{\partial #1}{\partial #2}}
}
\NewDocumentCommand \deln {mmm}
{
    \mathinner{\frac{\partial^#3 #1}{\partial #2 ^#3}}
}
\ExplSyntaxOff

\usepackage[
  math-style=ISO,
  bold-style=ISO,
  sans-style=italic,
  nabla=upright,
  partial=upright,
  warnings-off={
    mathtools-colon,
    mathtools-overbracket,
  },
]{unicode-math}

\setmathfont{Latin Modern Math}
\setmathfont{XITS Math}[range={scr, bfscr}]
\setmathfont{XITS Math}[range={cal, bfcal}, StylisticSet=1]

\usepackage[
version=4,
math-greek=default,
text-greek=default,
]{mhchem}

\usepackage[
  locale=DE,
  separate-uncertainty=true,
  per-mode=reciprocal,
  output-decimal-marker={,},
]{siunitx}

\usepackage[autostyle]{csquotes} %richtige Anführungszeichen

\usepackage{xfrac}

\usepackage{float}

\floatplacement{figure}{htbp}

\floatplacement{table}{htbp}

\usepackage[ %floats innerhalb einer section halten
  section,   %floats innerhalb er section halten
  below,     %unterhalb der Section aber auf der selben Seite ist ok
]{placeins}

\usepackage[
  labelfont=bf,
  font=small,
  width=0.9\textwidth,
]{caption}

\usepackage{subcaption} %subfigure, subtable, subref

\usepackage{graphicx}

\usepackage{ulem}

\usepackage{color}

\usepackage{grffile}

\usepackage{booktabs}

\sisetup{separate-uncertainty=true}

\usepackage{microtype} %Verbesserungen am Schriftbild

\usepackage[
backend=biber,
]{biblatex}

\addbibresource{../lit.bib}

\usepackage[ %Hyperlinks im Dokument
  german,
  unicode,
  pdfusetitle,
  pdfcreator={},
  pdfproducer={},
]{hyperref}

\usepackage{bookmark}

\usepackage[shortcuts]{extdash}


\usepackage{physics}
\allowdisplaybreaks

\begin{document}
    \title{Physik IV Übungsblatt 7}
    \author{  
    Tobias Rücker\\
    \texorpdfstring{\href{mailto:tobias.ruecker@tu-dortmund.de}{tobias.ruecker@tu-dortmund.de}
    \and}{,} 
    Paul Störbrock\\
    \texorpdfstring{\href{mailto:paul.stoerbrock@tu-dortmund.de}{paul.stoerbrock@tu-dortmund.de}}{}
    }
\maketitle
\center{\Large Abgabegruppe: \textbf{4H}}
\thispagestyle{empty}

\newpage
\tableofcontents
\thispagestyle{empty}
\newpage

\setcounter{page}{1}

\section{Aufgabe 1}

    \begin{figure}[H]
        \centering
        \includegraphics[width=\textwidth]{images/Aufgabe1.jpg}
        \label{fig:1}
    \end{figure}

    \subsection{a)}
    \begin{align}
    \int \dif{\omega} Y_{lm}^* (\theta, \phi) Y_{l'm'}(\theta , \phi)\\
    &=\int \dif{\omega} \frac{1}{\sqrt{2 \pi}} N_{lm} P_{lm} (\cos \theta) e^{-im\phi} 
    \frac{1}{\sqrt{2 \pi}} N_{l'm'} P_{l'm'}(\cos \theta) e^{im' \phi}\\
    &=\int \dif{\omega} \frac{1}{2 \pi} N_{lm} N_{l'm'} e^{i(m'-m)\phi} P_{lm} P_{l'm'}\\
    &=\int_0^{\pi} \int_0^{2 \pi} \frac{1}{2 \pi} N_{lm} N_{l'm'} \sin \theta e^{i(m'-m)\phi} P_{lm} P_{l'm'}
    \dif{\phi} \dif{\theta}\\
    &=\int_0^{\pi} \frac{1}{2\pi} N_{lm} N_{l'm'} \sin \theta P_{lm} P_{l'm'} \frac{1}{i(m'-m)} 
    \underbrace{(e^{i(m'-m)2 \pi}-1 )}_{=0} \dif{\theta} \\
    \intertext{
        für alle $m \ne m'$ 
        Fall $m=m'$
    }
    &=\int_0^{\pi} \int_0^{2 \pi} \frac{1}{2 \pi} N_{lm} N_{l'm} P_{lm} P_{l'm} \sin \theta e^{i(m-m)\phi} \dif{\phi} \dif{\theta}\\
    &= \int_0^{\pi} N_{lm} N_{l'm} P_{lm} P_{l'm} \sin \theta \dif{\theta} \\
    \Rightarrow &\int_0^{2\pi} \frac{1}{2 \pi} \frac{1}{2\pi} e^{i(m'-m)\phi} = \delta _{mm'}\\
    &=\int_0^{\pi} N_{lm} N_{l'm} P_{lm} P_{l'm} \underbrace{\sin \theta}_{-\frac{\dif{\cos \theta}}{\dif{\theta}}} \dif{\theta}\\
    &= \int_{\pi}^0 N_{lm} N_{l'm} P_{lm}(\cos \theta) P_{l'm} (\cos \theta) \dif{\cos \theta}\\
    \intertext{
        Substitution 
    }    
    x&= \cos \theta\\
    \frac{\dif{x}}{\dif{\cos \theta}} &=1\\
    \intertext{
        weiter
    }
    &= \int _{-1}^1 N_{lm} N_{lm'} P_{lm}(x) P_{l'm}(x) \dif{x}\\
    &= N_{lm} N_{l'm} \frac{2}{2l+1} \frac{(l+m)!}{(l-m)!} \delta _{ll'} \\
    &=\begin{cases}
    l=l' \;1\\
    l \ne l' \;0
    \end{cases}\\
    \Rightarrow \int \dif{\omega} Y_{l'm'}^* Y_{lm} = \delta_{ll'} \delta _{mm'}\\
    \sum_{l=0}^{\infty} \sum_{m=-l}^l Y_{lm}^* (\theta ' , \phi ') Y_{lm} (\theta , \phi) = \delta (\phi - \phi ') \delta (\cos \theta - \cos \theta ')\\
    \end{align}

    \subsection{b)}
    \begin{align}
    \intertext{
        \flushleft{Die\,}\justifying Kugelflächenfunktionen bilden ein orthonormales Funktionensystem. Durch ihre Vollständigkeit 
        bilden sie einen Satz von quadratintegrablen Funktionen im $L^2$ und können alle anderen Funktionen in $L^2$  
    }
    c_{lm} &= \int_0^{2 \pi} \int_0^\pi Y_{lm}^* f(\theta, \phi) \sin \theta \dif{\theta} \dif{\phi} \\
    \end{align}

    \subsection{c)}
    \begin{align}
        c_{lm} &= \int_0^{2 \pi} \int_0^\pi Y_{lm}^* f(\theta, \phi) \sin \theta \dif{\theta} \dif{\phi} \\
        \intertext{
            Transformation:
        }
        \phi ' &= \phi + \alpha \\
        \phi &= \phi ' - \alpha \\
        \frac{\dif{\phi}}{\dif{\phi '}} &=1\\
        \intertext{
            weiter
        }
        c_{lm} &= \int_0^{2 \pi} \int_0^{\pi} Y_{lm}^* (\theta , \phi '-\alpha) f(\theta , \phi '  - \alpha) \sin \theta \dif{\theta} \dif{\phi '}\\
        &= \int_0^{2 \pi} \int_0^{\pi} \frac{1}{\sqrt{2 \pi}} N_{lm} P_{lm} (\cos \theta) e^{-i m (\phi ' - \alpha)} d_{lm} \frac{1}{\sqrt{2 \pi}} N_{lm} P_{lm} (\cos{\theta}) e^{im(\phi ' + \alpha)}\\
        &= \int_0^{2 \pi} \frac{1}{2 \pi} \underbrace{\int_0^{\pi}  (P_{lm} (\cos \theta))^2}_{=1} e^{im 2 \alpha} d_{lm}\\
        &= e^{im2 \alpha} d_{lm}\\
        d_{lm} &= e^{-im2 \alpha} c_{lm}\\
        \abs{c_{lm}} &= \abs{e^{im 3 \alpha} c_{lm}} \\
        &=\sqrt{(e^{-im2 \alpha} d_{lm}^*)(e^{im2 \alpha} d_{lm})} \\
        \abs{c_{lm}} &= \abs{d_{lm}}
        \intertext{
            \flushleft{Die\,}\justifying Beträge bleiben bei der Transformation erhalten, während sich die normalen Koeffizienten um 
            eine Phase $\alpha$ unterscheiden.
        }
    \end{align}



\section{Aufgabe 2}

    \begin{figure}[H]
        \centering
        \includegraphics[width=\textwidth]{images/Aufgabe2.jpg}
        \label{fig:2}
    \end{figure}

    \subsection{a)}

    \subsection{b)}

    \subsection{c)}

    \subsection{d)}

    \subsection{e)}


\section{Aufgabe 3}

    \begin{figure}[H]
        \centering
        \includegraphics[width=\textwidth]{images/Aufgabe3.jpg}
        \label{fig:3}
    \end{figure}

    \subsection{a)}

    \subsection{b)}

    \subsection{c)}

    \subsection{d)}





\end{document}