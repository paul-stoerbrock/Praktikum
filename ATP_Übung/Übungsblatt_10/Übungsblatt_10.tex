\documentclass[
  captions=tableheading,
  bibliography=totoc, 
  titepage=firstiscover,
]{scrartcl}

\usepackage{blindtext} %neuer input

\usepackage{longtable} % Tabellen über mehrere Seiten

\usepackage[utf8]{inputenc} %neuer input

\usepackage{scrhack}

\usepackage[aux]{rerunfilecheck} %Warnung falls nochmal kompiliert werden muss

\usepackage{fontspec} %Fonteinstellungen

\recalctypearea{}

\usepackage[main=ngerman]{babel} %deutsche Spracheinstellung

\usepackage{ragged2e} %neuer input

\usepackage{amsmath, nccmath}

\usepackage{amssymb} %viele mathe Symbole

\usepackage{mathtools} %Erweiterungen für amsmath

\usepackage{MnSymbol}


\DeclarePairedDelimiter{\abs}{\lvert}{\rvert}
\DeclarePairedDelimiter{\norm}{\lVert}{\rVert}

\DeclarePairedDelimiter{\bra}{\langle}{\rvert}
\DeclarePairedDelimiter{\ket}{\lvert}{\rangle}

\DeclarePairedDelimiterX{\braket}[2]{\langle}{\rangle}{
#1 \delimsize| #2
}

\usepackage{expl3}
\usepackage{xparse}
\NewDocumentCommand \dif {m}
{
\mathinner{\symup{d} #1}
}

\NewDocumentCommand \del {mm}
{
    \mathinner{\frac{\partial #1}{\partial #2}}
}
\NewDocumentCommand \deln {mmm}
{
    \mathinner{\frac{\partial^#3 #1}{\partial #2 ^#3}}
}
\ExplSyntaxOff

\usepackage[
  math-style=ISO,
  bold-style=ISO,
  sans-style=italic,
  nabla=upright,
  partial=upright,
  warnings-off={
    mathtools-colon,
    mathtools-overbracket,
  },
]{unicode-math}

\setmathfont{Latin Modern Math}
\setmathfont{XITS Math}[range={scr, bfscr}]
\setmathfont{XITS Math}[range={cal, bfcal}, StylisticSet=1]

\usepackage[
version=4,
math-greek=default,
text-greek=default,
]{mhchem}

\usepackage[
  locale=DE,
  separate-uncertainty=true,
  per-mode=reciprocal,
  output-decimal-marker={,},
]{siunitx}

\usepackage[autostyle]{csquotes} %richtige Anführungszeichen

\usepackage{xfrac}

\usepackage{float}

\floatplacement{figure}{htbp}

\floatplacement{table}{htbp}

\usepackage[ %floats innerhalb einer section halten
  section,   %floats innerhalb er section halten
  below,     %unterhalb der Section aber auf der selben Seite ist ok
]{placeins}

\usepackage[
  labelfont=bf,
  font=small,
  width=0.9\textwidth,
]{caption}

\usepackage{subcaption} %subfigure, subtable, subref

\usepackage{graphicx}

\usepackage{ulem}

\usepackage{color}

\usepackage{grffile}

\usepackage{booktabs}

\sisetup{separate-uncertainty=true}

\usepackage{microtype} %Verbesserungen am Schriftbild

\usepackage[
backend=biber,
]{biblatex}

\addbibresource{../lit.bib}

\usepackage[ %Hyperlinks im Dokument
  german,
  unicode,
  pdfusetitle,
  pdfcreator={},
  pdfproducer={},
]{hyperref}

\usepackage{bookmark}

\usepackage[shortcuts]{extdash}



\begin{document}
    \title{ATP Übungsblatt 9}
    \author{  
    Tobias Rücker\\
    \texorpdfstring{\href{mailto:tobias.ruecker@tu-dortmund.de}{tobias.ruecker@tu-dortmund.de}
    \and}{,} 
    Paul Störbrock\\
    \texorpdfstring{\href{mailto:paul.stoerbrock@tu-dortmund.de}{paul.stoerbrock@tu-dortmund.de}}{}
    }
\maketitle
\center{\Large Abgabegruppe: \textbf{Mittw. 10-12 Uhr}}
\thispagestyle{empty}

\newpage
\tableofcontents
\thispagestyle{empty}
\newpage

\setcounter{page}{1}

\section{Aufgabe 28}

\begin{figure}[H]
    \centering
    \includegraphics[width=\textwidth]{images/Aufgabe28.jpg}
\end{figure}

\subsection{a)}

    \flushleft{Das\;}\justifying Hubble-Gesetz lautet wie folgt:
    \begin{align*}
        z &= H \cdot \frac{d}{c}
        \intertext{
            \flushleft{wobei\;}\justifying hier die Hubble-Konstante $H$ mit $H_0 \approx 70 \,\text{km s$^{-1}$Mpc$^{-1}$}$ genähert wird.
            $z$ beschreibt die Rotverschiebung und ist hier ungefähr gleich $0.057$, $d$ ist der Abstand und $c$ die Lichtgeschwindigkeit im Vakuum.
            Wird nach $d$ umgeformt, ergibt sich für den Abstand:
        }
        d &= \frac{zc}{H_0} \approx \text{244.117$\,$Mpc}
    \end{align*}
    \flushleft{Die\;}\justifying Näherung der Hubble-Konstante ist hier gerechtfertig, da das Licht <1 milliarden ly an Strecke zurückgelegt hat. Die Hubble-Konstante hat sich
    wahrscheinlich nur im frühen Universum (ca. vor 14 mio Jahren) verändert, da sich das Universum zu der Zeit sehr schnell ausgebreitet hat. Da das Licht nicht aus dem frühen 
    Universum stammt, ist die Näherung der Hubble-Konstante hier gerechtfertigt.

\subsection{b)}

    \flushleft{Das\;}\justifying Radiospektrum von Cygnus A ist durch das Potenzgesetz 
    \begin{align*}
        F_{\nu} &\approx \nu^{-\alpha}
        \intertext{
            \flushleft{gegeben\;}\justifying mit $\alpha = 0.08$ und dem Frequenzintervall $10^7\,\text{Hz} \leq \nu \leq 3\cdot 10^9$. Außerdem beträgt $F_{\nu}\approx 1255\,\text{Jy}$ 
            bei einer Frequenz von $\nu=1400\,\text{\SI{}{\mega\hertz}}$.\\
            Um die gesamte Radio-Leuchtkraft bestimmen zu können, muss die Proportionalitätskonstante des oben genannten Potenzgesetzes bestimmt werden:
        }
        F_{\nu} &= Const\cdot \nu^{-\alpha}\\
        \Leftrightarrow Const &= \frac{F_{\nu}}{\nu^{-\alpha}} = 2.603\cdot 10^{-16}\,\text{\SI{}{\watt\per\hertz\squared\meter\squared}}
        \intertext{
            \flushleft{Um\;}\justifying die Energieflussdichte im Abstand zur Erde bestimmen zu können, muss die spektrale Flussdichte über das Frequenzintervall integriert werden:
        }
        f &= \int_{\nu_1}^{\nu_2} Const \cdot \nu^{-\alpha} \mathrm{d}\nu = Const \int_{\nu_1}^{\nu_2} \nu^{-\alpha} \mathrm{d}\nu\\
        &= Const \left[\frac{1}{\alpha+1} \nu^{-\alpha+1} \right]_{\nu_1}^{\nu_2}\\
        &= \frac{Const}{\alpha+1} \left( \nu_2^{-\alpha+1}-\nu_1^{-\alpha+1} \right)\\
        &= 7.735\cdot 10^{-15}\,\text{\SI{}{\watt\per\meter\squared}}
        \intertext{
            \flushleft{Unter\;}\justifying der Annahme einer isotropen Abstrahlung in alle Richtungen kann die Leuchtkraft $L$ von Cygnus A wie folgt bestimmt werden:
        }
        L &= 4\pi R^2f
        \intertext{
            \flushleft{Der\;}\justifying Radius $R$ ist hier der Abstand zwischen der Galaxie und der Erde $d$. Daraus ergibt sich die Leuchtkraft:
        }
        L_{\text{Cygnus}} &= 5.515 \cdot 10^{36}\,\text{\SI{}{\watt}}
        \intertext{
            \flushleft{Die\;}\justifying Radio-Leuchtkraft der Galaxie Cygnus A ist somit vergleichbar mit der Leuchtkraft der Milchstraße $L\approx 5\cdot 10^{36}$\SI{}{\watt}.
        }
    \end{align*}

\section{Aufgabe 29}

\begin{figure}[H]
    \centering
    \includegraphics[width=\textwidth]{images/Aufgabe29.jpg}
\end{figure}

\subsection{a)}

\subsection{b)}

\subsection{c)}

\subsection{d)}

\section{Aufgabe 30}

\begin{figure}[H]
    \centering
    \includegraphics[width=\textwidth]{images/Aufgabe30a.jpg}
\end{figure}

\begin{figure}[H]
    \centering
    \includegraphics[width=\textwidth]{images/Aufgabe30b.jpg}
\end{figure}

\subsection{a)}

\begin{figure}[H]
    \centering
    \includegraphics[width=\textwidth]{images/30a.jpg}
\end{figure}

\subsection{b)}

\end{document}