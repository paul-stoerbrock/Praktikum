\documentclass[
  captions=tableheading,
  bibliography=totoc, 
  titepage=firstiscover,
]{scrartcl}

\usepackage{blindtext} %neuer input

\usepackage{longtable} % Tabellen über mehrere Seiten

\usepackage[utf8]{inputenc} %neuer input

\usepackage{scrhack}

\usepackage[aux]{rerunfilecheck} %Warnung falls nochmal kompiliert werden muss

\usepackage{fontspec} %Fonteinstellungen

\recalctypearea{}

\usepackage[main=ngerman]{babel} %deutsche Spracheinstellung

\usepackage{ragged2e} %neuer input

\usepackage{amsmath, nccmath}

\usepackage{amssymb} %viele mathe Symbole

\usepackage{mathtools} %Erweiterungen für amsmath

\usepackage{MnSymbol}


\DeclarePairedDelimiter{\abs}{\lvert}{\rvert}
\DeclarePairedDelimiter{\norm}{\lVert}{\rVert}

\DeclarePairedDelimiter{\bra}{\langle}{\rvert}
\DeclarePairedDelimiter{\ket}{\lvert}{\rangle}

\DeclarePairedDelimiterX{\braket}[2]{\langle}{\rangle}{
#1 \delimsize| #2
}

\usepackage{expl3}
\usepackage{xparse}
\NewDocumentCommand \dif {m}
{
\mathinner{\symup{d} #1}
}

\NewDocumentCommand \del {mm}
{
    \mathinner{\frac{\partial #1}{\partial #2}}
}
\NewDocumentCommand \deln {mmm}
{
    \mathinner{\frac{\partial^#3 #1}{\partial #2 ^#3}}
}
\ExplSyntaxOff

\usepackage[
  math-style=ISO,
  bold-style=ISO,
  sans-style=italic,
  nabla=upright,
  partial=upright,
  warnings-off={
    mathtools-colon,
    mathtools-overbracket,
  },
]{unicode-math}

\setmathfont{Latin Modern Math}
\setmathfont{XITS Math}[range={scr, bfscr}]
\setmathfont{XITS Math}[range={cal, bfcal}, StylisticSet=1]

\usepackage[
version=4,
math-greek=default,
text-greek=default,
]{mhchem}

\usepackage[
  locale=DE,
  separate-uncertainty=true,
  per-mode=reciprocal,
  output-decimal-marker={,},
]{siunitx}

\usepackage[autostyle]{csquotes} %richtige Anführungszeichen

\usepackage{xfrac}

\usepackage{float}

\floatplacement{figure}{htbp}

\floatplacement{table}{htbp}

\usepackage[ %floats innerhalb einer section halten
  section,   %floats innerhalb er section halten
  below,     %unterhalb der Section aber auf der selben Seite ist ok
]{placeins}

\usepackage[
  labelfont=bf,
  font=small,
  width=0.9\textwidth,
]{caption}

\usepackage{subcaption} %subfigure, subtable, subref

\usepackage{graphicx}

\usepackage{ulem}

\usepackage{color}

\usepackage{grffile}

\usepackage{booktabs}

\sisetup{separate-uncertainty=true}

\usepackage{microtype} %Verbesserungen am Schriftbild

\usepackage[
backend=biber,
]{biblatex}

\addbibresource{../lit.bib}

\usepackage[ %Hyperlinks im Dokument
  german,
  unicode,
  pdfusetitle,
  pdfcreator={},
  pdfproducer={},
]{hyperref}

\usepackage{bookmark}

\usepackage[shortcuts]{extdash}



\begin{document}
    \title{ATP Übungsblatt 1}
    \author{  
    Tobias Rücker\\
    \texorpdfstring{\href{mailto:tobias.ruecker@tu-dortmund.de}{tobias.ruecker@tu-dortmund.de}
    \and}{,} 
    Paul Störbrock\\
    \texorpdfstring{\href{mailto:paul.stoerbrock@tu-dortmund.de}{paul.stoerbrock@tu-dortmund.de}}{}
    }
\maketitle
\center{\Large Abgabegruppe: \textbf{Mittw. 10-12 Uhr}}
\thispagestyle{empty}

\newpage
\tableofcontents
\thispagestyle{empty}
\newpage

\setcounter{page}{1}


\section{Aufagabe 1}

\begin{figure}[H]
    \centering
    \includegraphics[width=0.75\textwidth]{images/Aufgabe_1.jpg}
    \label{fig:1}
\end{figure}

\subsection{a)}



\subsection{b)}



\subsection{c)}





\section{Aufgabe 2}

\begin{figure}[H]
    \centering
    \includegraphics[width=0.75\textwidth]{images/Aufgabe_2a.jpg}
    \label{fig:2}
\end{figure}

\begin{figure}[H]
    \centering
    \includegraphics[width=0.75\textwidth]{images/Aufgabe_2bcd.jpg}
    \label{fig:2}
\end{figure}

\subsection{a)}

    \begin{align}
        \intertext{\flushleft{Das\;}\justifying Stefan-Boltzmann-Gesetz lautet
        $P = \sigma \cdot A \cdot T^4$, wobei P die Luminosität der Schwarzkörpers ist.
        }
    \end{align}

        \begin{table}[H]
        \centering
        \begin{tabular}{l c}
            \toprule
                $T_{\text{Sonne}}$      & \input{T_S.tex}\\
                $R_{\text{Sonne}}$      & \input{R_S.tex}\\
                $R_{\text{Erde}}$       & \input{R_E.tex}\\
            \bottomrule
        \end{tabular}
        \caption{Messwerte}
        \label{tab:2a}
        \end{table}
    
    \begin{align}
        \intertext{\flushleft{Mit\;}\justifying der Formel lässt 
        sich die Oberfläche der Sonne und der Erde bestimmen:
        }
        O_{Sonne} &= \text{\input{A_S.tex}}\\
        O_{Erde} &= \text{input{A_E.tex}}\\
        \intertext{\flushleft{Die\;}\justifying Luminosität der Sonne beträgt:
        }
        P_{Sonne} &= \text{\input{P_S.tex}}
    \end{align}




\subsection{b)}



\subsection{c)}



\subsection{d)}



\section{Aufgabe 3}

\begin{figure}[H]
    \centering
    \includegraphics[width=0.75\textwidth]{images/Aufgabe_3.jpg}
    \label{fig:3}
\end{figure}

\subsection{a)}

\flushleft{Die\,}\justifying Wasserstofflinie der Wellenlänge $\lambda = \SI{21}{\centi\meter} $ ensteht durch den
Hyperfeinstrukturübergang des Wasserstoffatoms. Bei einem Wasserstoff können
die Spins entweder parallel oder entgegen gesetzt zueinander ausgerichtet sein.
Dabei ist die parallele Ausrichtung der enretisch höhere Zustand. Bei einem
Übergang von dem parallelen zum entgegengesetzten Zustand dem "Spin-Flip" wird
die H1 Linie ausgesendet.


\subsection{b)}

\flushleft{Die\,}\justifying Frequenz der H1-Wellenlänge beträgt:
\begin{align*}
    f_{H1}=\text{\input{f_H1.tex}}
\end{align*}
und die Energie zur Wellenlänge $\lambda = \SI{21}{\centi\meter} $ beträgt
\begin{align}
    E_{H1}=\text{\input{E_H1.tex}}
\end{align}


\subsection{c)}

\flushleft{Die\,}\justifying Wellenlängen der Wasserstofflinie mögen fest sein, allerdings bewegen sich
die Quellen, wodurch der Dopplereffekt auftritt. Lichtwellen vom z.B. sichbaren Spektrum deren Quellen sich vom
Beobachter entfernen werden ins rote verschoben, während Quellen, die sich auf den
Beobachter zubewegen ins blaue verschoben werden. Dadurch werden auch andere Wellenlängen  
beobachtet. 


\subsection{d)}



\end{document}