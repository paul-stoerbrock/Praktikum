\documentclass[
  captions=tableheading,
  bibliography=totoc, 
  titepage=firstiscover,
]{scrartcl}

\usepackage{blindtext} %neuer input

\usepackage{longtable} % Tabellen über mehrere Seiten

\usepackage[utf8]{inputenc} %neuer input

\usepackage{scrhack}

\usepackage[aux]{rerunfilecheck} %Warnung falls nochmal kompiliert werden muss

\usepackage{fontspec} %Fonteinstellungen

\recalctypearea{}

\usepackage[main=ngerman]{babel} %deutsche Spracheinstellung

\usepackage{ragged2e} %neuer input

\usepackage{amsmath, nccmath}

\usepackage{amssymb} %viele mathe Symbole

\usepackage{mathtools} %Erweiterungen für amsmath

\usepackage{MnSymbol}


\DeclarePairedDelimiter{\abs}{\lvert}{\rvert}
\DeclarePairedDelimiter{\norm}{\lVert}{\rVert}

\DeclarePairedDelimiter{\bra}{\langle}{\rvert}
\DeclarePairedDelimiter{\ket}{\lvert}{\rangle}

\DeclarePairedDelimiterX{\braket}[2]{\langle}{\rangle}{
#1 \delimsize| #2
}

\usepackage{expl3}
\usepackage{xparse}
\NewDocumentCommand \dif {m}
{
\mathinner{\symup{d} #1}
}

\NewDocumentCommand \del {mm}
{
    \mathinner{\frac{\partial #1}{\partial #2}}
}
\NewDocumentCommand \deln {mmm}
{
    \mathinner{\frac{\partial^#3 #1}{\partial #2 ^#3}}
}
\ExplSyntaxOff

\usepackage[
  math-style=ISO,
  bold-style=ISO,
  sans-style=italic,
  nabla=upright,
  partial=upright,
  warnings-off={
    mathtools-colon,
    mathtools-overbracket,
  },
]{unicode-math}

\setmathfont{Latin Modern Math}
\setmathfont{XITS Math}[range={scr, bfscr}]
\setmathfont{XITS Math}[range={cal, bfcal}, StylisticSet=1]

\usepackage[
version=4,
math-greek=default,
text-greek=default,
]{mhchem}

\usepackage[
  locale=DE,
  separate-uncertainty=true,
  per-mode=reciprocal,
  output-decimal-marker={,},
]{siunitx}

\usepackage[autostyle]{csquotes} %richtige Anführungszeichen

\usepackage{xfrac}

\usepackage{float}

\floatplacement{figure}{htbp}

\floatplacement{table}{htbp}

\usepackage[ %floats innerhalb einer section halten
  section,   %floats innerhalb er section halten
  below,     %unterhalb der Section aber auf der selben Seite ist ok
]{placeins}

\usepackage[
  labelfont=bf,
  font=small,
  width=0.9\textwidth,
]{caption}

\usepackage{subcaption} %subfigure, subtable, subref

\usepackage{graphicx}

\usepackage{ulem}

\usepackage{color}

\usepackage{grffile}

\usepackage{booktabs}

\sisetup{separate-uncertainty=true}

\usepackage{microtype} %Verbesserungen am Schriftbild

\usepackage[
backend=biber,
]{biblatex}

\addbibresource{../lit.bib}

\usepackage[ %Hyperlinks im Dokument
  german,
  unicode,
  pdfusetitle,
  pdfcreator={},
  pdfproducer={},
]{hyperref}

\usepackage{bookmark}

\usepackage[shortcuts]{extdash}



\begin{document}
    \title{ATP Übungsblatt 7}
    \author{  
    Tobias Rücker\\
    \texorpdfstring{\href{mailto:tobias.ruecker@tu-dortmund.de}{tobias.ruecker@tu-dortmund.de}
    \and}{,} 
    Paul Störbrock\\
    \texorpdfstring{\href{mailto:paul.stoerbrock@tu-dortmund.de}{paul.stoerbrock@tu-dortmund.de}}{}
    }
\maketitle
\center{\Large Abgabegruppe: \textbf{Mittw. 10-12 Uhr}}
\thispagestyle{empty}

\newpage
\tableofcontents
\thispagestyle{empty}
\newpage

\setcounter{page}{1}


\section{Aufgabe 19}

    \begin{figure}[H]
        \centering
        \includegraphics[width=\textwidth]{images/Aufgabe19a.jpg}
        \label{fig:1}
    \end{figure}

    \begin{figure}[H]
        \centering
        \includegraphics[width=\textwidth]{images/Aufgabe19b.jpg}
        \label{fig:2}
    \end{figure}

\subsection{a)}

\subsection{b)}

\subsection{c)}

\subsection{d)}

\subsection{e)}

\subsection{f)}


\section{Aufgabe 20}

    \begin{figure}[H]
        \centering
        \includegraphics[width=\textwidth]{images/Aufgabe20.jpg}
        \label{fig:3}
    \end{figure}


\subsection{a)}
Beim Cerenkov-Effekt bewegt sich ein Teilchen in einem Medium mit Brechungsindex n schneller als das Licht in diesem Medium.
Die durch das Teilchen erzeugten Kugelwellen entlang der Trajektorie überlagern sich, sodass eine Kegelförmige Wellenfront
entsteht.
\begin{figure}
    \centering
    \includegraphics[width=0.5\textwidth]{images/kegel.jpg}
\end{figure}

\subsection{b)}
\begin{figure}
\centering
\includegraphics[width=0.5\textwidth]{images/winkel.jpg}
\end{figure}
\begin{align}
\theta = \arcsin \left( \frac{c}{n \beta} \right)
\end{align}
\subsection{c)}
\begin{align}
\beta &= \frac{c}{n}\\
E&= \gamma m_0 c^2\\
&= \frac{m_e c^2}{\sqrt{1-\frac{\beta ^2}{c^2}}}\\
&= \frac{m_e c^2}{\sqrt{1-\frac{1}{n^2}}}\\
\intertext{
    Für Luft
}
E = \input{luft.tex}
\intertext{
    Für Wasser
}
E =\input{wasser.tex}
\end{align}

\subsection{d)}
Die Vorteile eines Teilchendetektors, der den Cherenkov-Effekt ausnutzt ist der, dass
zum einen das dieser sensitiv für die Einfallrichtung ist. Das heißt man kann
bestimmen aus welcher Richtung Leptonen-Schauer kommen und damit Quellen dieser finden.
Zudem müssen die Signale nicht im Detektor landen, da diese Kegel ein ganzes Stück bewegen können. 
Dadurch ist der messbare Bereich ein ganzes Stück größer als der eigentliche Detektor.


\subsection{e)}



\section{Aufgabe 21}

    \begin{figure}[H]
        \centering
        \includegraphics[width=\textwidth]{images/Aufgabe21.jpg}
        \label{fig:4}
    \end{figure}


\subsection{a)}

\subsection{b)}

\subsection{c)}

\subsection{d)}

\subsection{e)}

\subsection{f)}




\end{document}