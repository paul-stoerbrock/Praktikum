\documentclass[
  captions=tableheading,
  bibliography=totoc, 
  titepage=firstiscover,
]{scrartcl}

\usepackage{blindtext} %neuer input

\usepackage{longtable} % Tabellen über mehrere Seiten

\usepackage[utf8]{inputenc} %neuer input

\usepackage{scrhack}

\usepackage[aux]{rerunfilecheck} %Warnung falls nochmal kompiliert werden muss

\usepackage{fontspec} %Fonteinstellungen

\recalctypearea{}

\usepackage[main=ngerman]{babel} %deutsche Spracheinstellung

\usepackage{ragged2e} %neuer input

\usepackage{amsmath, nccmath}

\usepackage{amssymb} %viele mathe Symbole

\usepackage{mathtools} %Erweiterungen für amsmath

\usepackage{MnSymbol}


\DeclarePairedDelimiter{\abs}{\lvert}{\rvert}
\DeclarePairedDelimiter{\norm}{\lVert}{\rVert}

\DeclarePairedDelimiter{\bra}{\langle}{\rvert}
\DeclarePairedDelimiter{\ket}{\lvert}{\rangle}

\DeclarePairedDelimiterX{\braket}[2]{\langle}{\rangle}{
#1 \delimsize| #2
}

\usepackage{expl3}
\usepackage{xparse}
\NewDocumentCommand \dif {m}
{
\mathinner{\symup{d} #1}
}

\NewDocumentCommand \del {mm}
{
    \mathinner{\frac{\partial #1}{\partial #2}}
}
\NewDocumentCommand \deln {mmm}
{
    \mathinner{\frac{\partial^#3 #1}{\partial #2 ^#3}}
}
\ExplSyntaxOff

\usepackage[
  math-style=ISO,
  bold-style=ISO,
  sans-style=italic,
  nabla=upright,
  partial=upright,
  warnings-off={
    mathtools-colon,
    mathtools-overbracket,
  },
]{unicode-math}

\setmathfont{Latin Modern Math}
\setmathfont{XITS Math}[range={scr, bfscr}]
\setmathfont{XITS Math}[range={cal, bfcal}, StylisticSet=1]

\usepackage[
version=4,
math-greek=default,
text-greek=default,
]{mhchem}

\usepackage[
  locale=DE,
  separate-uncertainty=true,
  per-mode=reciprocal,
  output-decimal-marker={,},
]{siunitx}

\usepackage[autostyle]{csquotes} %richtige Anführungszeichen

\usepackage{xfrac}

\usepackage{float}

\floatplacement{figure}{htbp}

\floatplacement{table}{htbp}

\usepackage[ %floats innerhalb einer section halten
  section,   %floats innerhalb er section halten
  below,     %unterhalb der Section aber auf der selben Seite ist ok
]{placeins}

\usepackage[
  labelfont=bf,
  font=small,
  width=0.9\textwidth,
]{caption}

\usepackage{subcaption} %subfigure, subtable, subref

\usepackage{graphicx}

\usepackage{ulem}

\usepackage{color}

\usepackage{grffile}

\usepackage{booktabs}

\sisetup{separate-uncertainty=true}

\usepackage{microtype} %Verbesserungen am Schriftbild

\usepackage[
backend=biber,
]{biblatex}

\addbibresource{../lit.bib}

\usepackage[ %Hyperlinks im Dokument
  german,
  unicode,
  pdfusetitle,
  pdfcreator={},
  pdfproducer={},
]{hyperref}

\usepackage{bookmark}

\usepackage[shortcuts]{extdash}



\begin{document}
    \title{ATP Übungsblatt 6}
    \author{  
    Tobias Rücker\\
    \texorpdfstring{\href{mailto:tobias.ruecker@tu-dortmund.de}{tobias.ruecker@tu-dortmund.de}
    \and}{,} 
    Paul Störbrock\\
    \texorpdfstring{\href{mailto:paul.stoerbrock@tu-dortmund.de}{paul.stoerbrock@tu-dortmund.de}}{}
    }
\maketitle
\center{\Large Abgabegruppe: \textbf{Mittw. 10-12 Uhr}}
\thispagestyle{empty}

\newpage
\tableofcontents
\thispagestyle{empty}
\newpage

\setcounter{page}{1}

\section{Aufgabe 16}

    \begin{figure}[H]
        \centering
        \includegraphics[width=\textwidth]{images/Aufgabe16a.jpg}
        \label{fig:1}
    \end{figure}

    \begin{figure}[H]
        \centering
        \includegraphics[width=\textwidth]{images/Aufgabe16b.jpg}
        \label{fig:2}
    \end{figure}

\subsection{a)}

\subsection{b)}



\section{Aufgabe 17}

    \begin{figure}[H]
        \centering
        \includegraphics[width=\textwidth]{images/Aufgabe17a.jpg}
        \label{fig:3}
    \end{figure}

    \begin{figure}[H]
        \centering
        \includegraphics[width=\textwidth]{images/Aufgabe17b.jpg}
        \label{fig:4}
    \end{figure}

\subsection{a)}

\subsection{b)}

\subsection{c)}





\section{Aufgabe 18}

    \begin{figure}[H]
        \centering
        \includegraphics[width=\textwidth]{images/Aufgabe18a.jpg}
        \label{fig:5}
    \end{figure}

    \begin{figure}[H]
        \centering
        \includegraphics[width=\textwidth]{images/Aufgabe18b.jpg}
        \label{fig:6}
    \end{figure}

\subsection{a)}

    \flushleft{Vor\;}\justifying Kohlhörsters Experiment, wurde angenommen, dass der Strahlungshintergrund ausschließlich von der Untergrundstrahlung der Erde ausgeht. 
    Demnach sollte der Strahlungshintergrund mit zunehmender Höhe abfallen. Kohlhörsters Ballonflug hingegen zeigte, dass die nicht der Fall ist. Sondern vielmehr, dass
    der Strahlungshintergrund bei zunehmender Höhe auf ein Vielfaches dessen steigt, was am Erdboden gemessen wird. Interessanterweise, wird bei einer relativ geringen 
    Höhe (ca.1000m) ein Abfall der Strahlung gemessen, was mit der vorherigen Ansicht, nämlich, dass die Untergrundstrahlung bei zunehmender Höhe abnimmt, übereinstimmt.\\
    Heutzutage wissen wir, dass die gemessene kosmische Strahlung bei größerer Höhe immer weiter zunimmt. Dies hat mit der Ionisation der Luft zu tun, die bei zunehmender
    Höhe an Dichte verliert, weshalb die kosmische Strahlung weniger abgeschiermt werden kann.

\subsection{b)}

\subsection{c)}

\subsection{d)}





\end{document}