\documentclass[
  captions=tableheading,
  bibliography=totoc, 
  titepage=firstiscover,
]{scrartcl}

\usepackage{blindtext} %neuer input

\usepackage{longtable} % Tabellen über mehrere Seiten

\usepackage[utf8]{inputenc} %neuer input

\usepackage{scrhack}

\usepackage[aux]{rerunfilecheck} %Warnung falls nochmal kompiliert werden muss

\usepackage{fontspec} %Fonteinstellungen

\recalctypearea{}

\usepackage[main=ngerman]{babel} %deutsche Spracheinstellung

\usepackage{ragged2e} %neuer input

\usepackage{amsmath, nccmath}

\usepackage{amssymb} %viele mathe Symbole

\usepackage{mathtools} %Erweiterungen für amsmath


\DeclarePairedDelimiter{\abs}{\lvert}{\rvert}
\DeclarePairedDelimiter{\norm}{\lVert}{\rVert}

\DeclarePairedDelimiter{\bra}{\langle}{\rvert}
\DeclarePairedDelimiter{\ket}{\lvert}{\rangle}

\DeclarePairedDelimiterX{\braket}[2]{\langle}{\rangle}{
#1 \delimsize| #2
}

\NewDocumentCommand \dif {m}
{
\mathinner{\symup{d} #1}
}


\usepackage[
  math-style=ISO,
  bold-style=ISO,
  sans-style=italic,
  nabla=upright,
  partial=upright,
  warnings-off={
    mathtools-colon,
    mathtools-overbracket,
  },
]{unicode-math}

\setmathfont{Latin Modern Math}
\setmathfont{XITS Math}[range={scr, bfscr}]
\setmathfont{XITS Math}[range={cal, bfcal}, StylisticSet=1]


\usepackage[
  locale=DE,
  separate-uncertainty=true,
  per-mode=reciprocal,
  output-decimal-marker={,},
]{siunitx}

\usepackage[autostyle]{csquotes} %richtige Anführungszeichen

\usepackage{xfrac}

\usepackage{float}

\floatplacement{figure}{htbp}

\floatplacement{table}{htbp}

\usepackage[ %floats innerhalb einer section halten
  section,   %floats innerhalb er section halten
  below,     %unterhalb der Section aber auf der selben Seite ist ok
]{placeins}

\usepackage[
  labelfont=bf,
  font=small,
  width=0.9\textwidth,
]{caption}

\usepackage{subcaption} %subfigure, subtable, subref

\usepackage{graphicx}

\usepackage{grffile}

\usepackage{booktabs}

\usepackage{microtype} %Verbesserungen am Schriftbild

\usepackage[
backend=biber,
]{biblatex}

\addbibresource{../lit.bib}

\usepackage[ %Hyperlinks im Dokument
  german,
  unicode,
  pdfusetitle,
  pdfcreator={},
  pdfproducer={},
]{hyperref}

\usepackage{bookmark}

\usepackage[shortcuts]{extdash}

%\usepackage{warpcol}


\begin{document}
    \title{V103 Biegung elastischer Stäbe}
    \author{  
    Tobias Rücker\\
    \texorpdfstring{\href{mailto:tobias.ruecker@tu-dortmund.de}{tobias.ruecker@tu-dortmund.de}
    \and}{,} 
    Paul Störbrock\\
    \texorpdfstring{\href{mailto:paul.stoerbrock@tu-dortmund.de}{paul.stoerbrock@tu-dortmund.de}}{}
    }
    \date{Durchführung: 03.12.2019, Abgabe: 10.12.2019\vspace{-4ex}}
\maketitle
\center{\Large Versuchsgruppe: \textbf{42}}
    
    \begin{abstract}
    \centering
        \textbf{Ziel:} 
    \end{abstract}

\newpage
\tableofcontents
\newpage

% Theorie %%%%%%%%%%%%%%%%%%%%%%%%%%%%%%%%%%%%%%%%%%%%%%%%%%%%%%%%%%%%%%%%%%%%%%%%%%%%%%%%%%%%%%%%%%%%%%%%%

\section{Theorie}\justifying
Ein Körper kann eine Volumenänderung, wenn an dessen Oberfläche eine Kraft angelegt
wird. Diese Kraft hat einen engen Bezug zur Fläche des Körpers und wird in der Physik
als Spannung $\sigma$ bezeichnet.\\
Ein Beispiel für eine Volumenveränderung stellt eine Einzelbiegung eines Stabes der Länge L dar.
Dabei wird ein Ende des Stabes fixiert, während auf dem anderen Ende des Stabes eine
Gewichtskraft über eine zusätzliche Masse wirkt. Dabei wird ein Teil des Stabes gestreckt,
ein zweiter gestaucht und ein dritter Teil bleibt gleich. Der Stab stellt dabei aufgrund der inneren
Kräfte und der elastische Eigenschaften eine konstante Auslenkung ein.
Bei kleinen Veränderungen
besteht ein linearer Zusammenhang zwischen der Spannung $\sigma$  und der Deformation $\sfrac{\delta x}{\Delta x} $
\begin{align}
    \sigma (y)=E \frac{\delta x}{\Delta x} \label{eq:1}.
\end{align}
Diese Kraft wird auch das Hooksche Gesetz genannt. Der Faktor E in der Formel wird
der Elastizitätmodul genannt und stellt eine Materialkonstate dar.
Das durch die Biegung beschriebene Drehmoment führt zu folgender Momentengleichung
\begin{align}
    E \frac{\symup{d}^2 \, D}{\symup{d}x^2}\int_Q y^2 \, \symup{d}q = F(L-x) \label{eq:2}
\end{align}
Dabei ist D die Durchbiegung des Stabes und 
\begin{align}
    I := \int_Q y^2 \, \symup{d}q(y) \label{eq:3}
\end{align}
das Flächenträgheitsmoment. \\
Aus Gleichung \eqref{eq:2} folgt für die Durchbiegung D 
\begin{align}
    D(x)=\frac{F}{2E\,I}\left(L \, x^2-\frac{x^3}{3}\right) \label{eq:4} (\text{für} \, 0\leq x \leq L).
\end{align}
Eine weitere mögliche Biegungsart stellt die Doppelbiegung dar. Bei dieser wird
der Stab an beiden Enden aufgelegt und lässt die Gewichtskraft auf die Stabmitte 
wirken. \\
Für die Beschreibung der Durchbiegung wird in diesem Fall zwischen den Bereichen
$ 0\leq x\leq \sfrac{L}{2} $ und $v\sfrac{L}{2} \leq x \leq L $ unterschieden.\\
Für den Bereich $0\leq x\leq \sfrac{L}{2}$ ergibt sich für D
\begin{align}
    D(x)=\frac{F}{48\, E\, I}(3L^2x-4x^3) \label{eq:5}
\end{align}
und für den anderen Bereich $\sfrac{L}{2} \leq x \leq L $ ergibt sich entsprechend
\begin{align}
    D(x)=\frac{F}{48\, E\, I}(4x^3 - 12Lx^2 +9L^2x-L^3) \label{eq:6}.
\end{align}

% Fehlerrechnung %%%%%%%%%%%%%%%%%%%%%%%%%%%%%%%%%%%%%%%%%%%%%%%%%%%%%%%%%%%%%%%%%%%%%%%%%%%%%%%%%%%%%%%%%%%%%%%%%%%

\section{Fehlerrechnung}\justifying
Für eventuell anfallende Fehler in der Auswertung werden folgende Formeln verwendet:
\begin{subequations}
\begin{align}
\intertext{Für den Mittelwert wird die Formel
}
    \overline{x} &= \frac{1}{N}\sum_{i=1}^{N} x_i \label{eq:7a}.
\intertext{verwendet. Der Fehler des Mittelwerts wird mit der Formel:
}
    \Delta\overline{x} &= \frac{1}{\sqrt{N}} \sqrt{\frac{1}{1-N} \sum_{i=1}^{N} (x_i - \overline{x})^2} \label{eq:7b},
\intertext{berechnet, und die Gaußsche Fehlerfortpflanzung wird wie folgt bestimmt:
}
    \Delta f &= \sqrt{\sum_{i=1}^{N} \left( \frac{\delta f}{\delta x_i} \right)^2 \cdot (\Delta x_i)^2} \label{eq:7c}
\intertext{Die für eine lineare Regression benötigte Ausgleichsgerade wird mit den folgenden Parametern berechnet:
}
    y &= m \cdot x + b \label{eq:7d} \\ 
    m &= \frac{\overline{xy} - \overline{x} \cdot \overline{y}}{\overline{x^2} - {\overline{x}}^2} \label{eq:7e}\\
    b &= \frac{\overline{y} \cdot \overline{x^2} - \overline{xy} \cdot \overline{x}}{\overline{x^2} - {\overline{x}}^2} \label{eq:7f}
\end{align}
\end{subequations}
\newpage

% Versuchsaufbau/Versuchsdurchführung %%%%%%%%%%%%%%%%%%%%%%%%%%%%%%%%%%%%%%%%%%%%%%%%%%%%%%%%%%%%%%%%%%%%%%%%%%%%%%%%%%%%%%%%%%%%%%%%%%%%%%%%%%%%%%%%%%%%%%%%%%%%%%%%%%%%%%%%%%%%%%%%%%%%%%

\section{Versuchsaufbau/Durchführung}\justifying

\flushleft{Benötigt\;}\justifying werden: \textit{ Eine ca. $\SI{60}{\centi\meter}$ lange, $\SI{530}{\gram}$ schwere, rechtreckige Kupfterstange, eine ca. $\SI{60}{\centi\meter}$ lange, runde
Aluminiumstange, eine Aperatur zur Biegung der Stäbe mit zwei Fixpunkten, zwei schiebbare Messuhren im $\SI{}{\micro\meter}$-Bereich, eine
$\SI{55}{\centi\meter}$ lange Skala für die Messuhren, eine Aufhängung ($\SI{19}{\gram}$), acht Gewichte ($4x$ ca. $\SI{1160}{\gram}$, 
$2x$ ca. $\SI{500}{\gram}$, $1x$ ca. $\SI{225}{\gram}$), ein Maßband, eine Waage.
}
\flushleft{Zuerst\;}\justifying wird die Kupferstange an einem der Fixpunkte eingespannt. Die Messuhren werden über der Stange auf der Skala platziert. Nun werden mit 
einer Messuhr 19 Messwerte der Durchbiegung in einem Abstand von $\SI{2.5}{\centi\meter}$ (von $\SI{2.5}{\centi\meter}$ bis $\SI{47.5}{\centi\meter}$) 
bestimmt. Nach den Messungen ohne Gewicht, werden zwei $\SI{500}{\gram}$ Gewichte an das lose Ende gehangen und die Messungen wiederholt. Anschließend
werden die Gewichte abgenommen und die Kupferstange an beiden Enden fixiert. Nun wird die Durchbiegung an 14 Stellen mit einem Abstand von 
$\SI{3}{\centi\meter}$ gemessen. Danach werden vier $\SI{1160}{\gram}$ Gewichte mittig an die Stange gehangen und die Messungen wiederholt. 

\flushleft{Der\;}\justifying obige Prozess wird für die Runde Aluminiumstange wiederholt. Zu Beginn wird die Aluminiumstange einseitig eingespannt. Die Durchbiegung
wird ohne Gewichte 19-mal in einem Abstand von $\SI{2.5}{\centi\meter}$ gemessen. Dann wird ein $\SI{500}{\gram}$ Gewicht an das lose Ende 
gehangen und die Messung wiederholt. Anschließend wird das Gewicht abgenommen und die Stange an beiden Seiten fixiert. Abschließend werden 
14 Messungen ohne Gewichte und 14 Messungen mit zwei $\SI{500}{\gram}$ Gewichten und einem $\SI{230}{\gram}$ Gewicht durchgeführt.

% Auswertung %%%%%%%%%%%%%%%%%%%%%%%%%%%%%%%%%%%%%%%%%%%%%%%%%%%%%%%%%%%%%%%%%%%%%%%%%%%%%%%%%%%%%%%%%%%%%%%%%%%%%%%%%%%%%%%%%%%%%%%%%%%%%%%%%%%%%%%%%%%

\section{Auswertung}\justifying

% Cu_ein --------------------------------------------------------------------------------------------------------------------------------

\subsection{Kupfer einfach fixiert}

\flushleft{Das\;}\justifying Flächenträgheitsmoment des Kupfterstabs beträgt:
\input{I_Cu.tex}

\begin{align}
    \text{Länge des Kupfterstabs:}
    \text{\input{l_Cu.tex}}\\
    \text{Breite des Kupfterstabs:}
    \text{\input{d_Cu.tex}}\\
    \text{Dicke des Kupfterstabs:}
    \text{\input{b_Cu.tex}}\\
    \text{Masse der Aufhängung:}
    \text{\input{m_aufhaeng.tex}}\\
    \text{Masse der Schraube:}
    \text{\input{m_schraube.tex}}
\end{align}

\begin{align}
    \text{Angehängte Massen der einfachen Biegung:}\\
    \text{\input{m_Cuein1.tex}}\\
    \text{\input{m_Cuein2.tex}}
\end{align}

\begin{table}[H]
    \centering
    \input{Cu_ein.tex}
    \caption{Messwerte der Kupferstange einfach fixiert}
    \label{tab:1}
\end{table}

\flushleft{Die\;}\justifying für die lineare Regression relevanten Parameter aus Graph \ref{fig:1} lauten:

m = \input{m_PlotCuein.tex}
b = \input{b_PlotCuein.tex}

\begin{figure}[H]
    \centering
    \includegraphics[width=\linewidth]{plotCuein.pdf}
    \caption{Graph Einzelbiegung}
    \label{fig:1}
\end{figure}

% Cu_dop --------------------------------------------------------------------------------------------------------------------------------
\subsection{Kupfer doppelt fixiert}

\flushleft{Der\;}\justifying Mittelpunkt, an dem die Gewichte bei der doppelt fixierten Biegung aufgehängt werden, liegt bei 
$\SI{27.5}{\centi\meter}$. Dieser wird nicht in den dazugehörigen Tabellen aufgeführt.

\begin{align}
    \text{Angehängte Massen der Doppelbiegung:}\\
    \text{\input{m_Cudop1.tex}}\\
    \text{\input{m_Cudop2.tex}}\\
    \text{\input{m_Cudop3.tex}}
\end{align}

\begin{table}[H]
    \centering
    \input{Cu_dop.tex}
    \caption{Messwerte der Kupferstange doppelt fixiert}
    \label{tab:2}
\end{table}

\flushleft{Die\;}\justifying für die lineare Regression relevanten Parameter aus Graph \ref{fig:2a} und \ref{fig:2b} lauten:

m(a) = \input{m_PlotCudopl.tex}
b(a) = \input{b_PlotCudopl.tex}

m(b) = \input{m_PlotCudopr.tex}
b(b) = \input{b_PlotCudopr.tex}

\begin{figure}[H]
\caption{Graph für Doppelbiegung, $Cu_{links}$ und $Cu_{rechts}$}
\label{fig:2}
\begin{subfigure}{0.495\linewidth}
    \centering
    \includegraphics[width=\linewidth]{plotCudopl.pdf}
    \caption{$Cu_{links}$}
    \label{fig:2a}
\end{subfigure}
\begin{subfigure}{0.495\linewidth}
    \centering
    \includegraphics[width=\linewidth]{plotCudopr.pdf}
    \caption{$Cu_{rechts}$}
    \label{fig:2b}
\end{subfigure}

\end{figure}

% Al_ein --------------------------------------------------------------------------------------------------------------------------------

\subsection{Aluminium einfach fixiert}

\flushleft{Das\;}\justifying Flächenträgheitsmoment des Aluminiumstabs beträgt:
\input{I_Al.tex}

\begin{align}
    \text{Länge des Kupfterstabs:}\\
    \text{\input{l_Al.tex}}\\
    \text{Durchmesser des Aluminiumstabs:}\\
    \text{\input{d_Al.tex}}\\
\end{align}

\begin{align}
    \text{Angehängte Masse der Einzelbiegung:}\\
    \text{\input{m_Alein1.tex}}
\end{align}

\begin{table}[H]
    \centering
    \input{Al_ein.tex}
    \caption{Messwerte der Aluminiumstange einfach fixiert}
    \label{tab:3}
\end{table}

\flushleft{Die\;}\justifying für die lineare Regression relevanten Parameter aus Graph \ref{fig:3} lauten:

m = \input{m_PlotAlein.tex}
b = \input{b_PlotAlein.tex}

\begin{figure}[H]
    \centering
    \includegraphics[width=\linewidth]{plotAlein.pdf}
    \caption{Graph Einzelbiegung}
    \label{fig:3}
\end{figure}

% Al_dop --------------------------------------------------------------------------------------------------------------------------------

\subsection{Aluminium doppelt fixiert}

\begin{align}
    \text{Angehängte Massen der Doppelbiegung:}\\
    \text{\input{m_Aldop1.tex}}\\
    \text{\input{m_Aldop2.tex}}\\
    \text{\input{m_Aldop3.tex}}
\end{align}

\begin{table}[H]
    \centering
    \input{Al_dop.tex}
    \caption{Messwerte der Aluminiumstange doppelt fixiert}
    \label{tab:4}
\end{table}

\flushleft{Die\;}\justifying für die lineare Regression relevanten Parameter aus Graph \ref{fig:4a} und \ref{fig:4b} lauten:

m(a) = \input{m_PlotAldopl.tex}
b(a) = \input{b_PlotAldopl.tex}

m(b) = \input{m_PlotAldopr.tex}
b(b) = \input{b_PlotAldopr.tex}


\begin{figure}[H]
\caption{Graph für Doppelbiegung, $Al_{links}$ und $Al_{rechts}$}
\label{fig:4}
\begin{subfigure}{0.495\linewidth}
    \centering
    \includegraphics[width=\linewidth]{plotAldopl.pdf}
    \caption{$Al_{links}$}
    \label{fig:4a}
\end{subfigure}
\begin{subfigure}{0.495\linewidth}
    \centering
    \includegraphics[width=\linewidth]{plotAldopr.pdf}
    \caption{$Al_{rechts}$}
    \label{fig:4b}
\end{subfigure}
\end{figure}

% Diskussion %%%%%%%%%%%%%%%%%%%%%%%%%%%%%%%%%%%%%%%%%%%%%%%%%%%%%%%%%%%%%%%%%%%%%%%%%%%%%%%%%%%%%%%%%%%%%%%%%%%%%%%%%%%%%%%%%%%%%%%%%%%%%%%%%%%%%%%%%%%

\section{Diskussion}\justifying

\newpage

\printbibliography
\end{document}