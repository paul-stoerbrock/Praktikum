\documentclass[
  captions=tableheading,
  bibliography=totoc, 
  titepage=firstiscover,
]{scrartcl}

\usepackage{blindtext} %neuer input

\usepackage{longtable} % Tabellen über mehrere Seiten

\usepackage[utf8]{inputenc} %neuer input

\usepackage{scrhack}

\usepackage[aux]{rerunfilecheck} %Warnung falls nochmal kompiliert werden muss

\usepackage{fontspec} %Fonteinstellungen

\recalctypearea{}

\usepackage[main=ngerman]{babel} %deutsche Spracheinstellung

\usepackage{ragged2e} %neuer input

\usepackage{amsmath, nccmath}

\usepackage{amssymb} %viele mathe Symbole

\usepackage{mathtools} %Erweiterungen für amsmath


\DeclarePairedDelimiter{\abs}{\lvert}{\rvert}
\DeclarePairedDelimiter{\norm}{\lVert}{\rVert}

\DeclarePairedDelimiter{\bra}{\langle}{\rvert}
\DeclarePairedDelimiter{\ket}{\lvert}{\rangle}

\DeclarePairedDelimiterX{\braket}[2]{\langle}{\rangle}{
#1 \delimsize| #2
}

\NewDocumentCommand \dif {m}
{
\mathinner{\symup{d} #1}
}


\usepackage[
  math-style=ISO,
  bold-style=ISO,
  sans-style=italic,
  nabla=upright,
  partial=upright,
  warnings-off={
    mathtools-colon,
    mathtools-overbracket,
  },
]{unicode-math}

\setmathfont{Latin Modern Math}
\setmathfont{XITS Math}[range={scr, bfscr}]
\setmathfont{XITS Math}[range={cal, bfcal}, StylisticSet=1]


\usepackage[
  locale=DE,
  separate-uncertainty=true,
  per-mode=reciprocal,
  output-decimal-marker={,},
]{siunitx}

\usepackage[autostyle]{csquotes} %richtige Anführungszeichen

\usepackage{xfrac}

\usepackage{float}

\floatplacement{figure}{htbp}

\floatplacement{table}{htbp}

\usepackage[ %floats innerhalb einer section halten
  section,   %floats innerhalb er section halten
  below,     %unterhalb der Section aber auf der selben Seite ist ok
]{placeins}

\usepackage[
  labelfont=bf,
  font=small,
  width=0.9\textwidth,
]{caption}

\usepackage{subcaption} %subfigure, subtable, subref

\usepackage{graphicx}

\usepackage{grffile}

\usepackage{booktabs}

\usepackage{microtype} %Verbesserungen am Schriftbild

\usepackage[
backend=biber,
]{biblatex}

\addbibresource{../lit.bib}

\usepackage[ %Hyperlinks im Dokument
  german,
  unicode,
  pdfusetitle,
  pdfcreator={},
  pdfproducer={},
]{hyperref}

\usepackage{bookmark}

\usepackage[shortcuts]{extdash}

%\usepackage{warpcol}


\begin{document}
    \title{V103 Biegung elastischer Stäbe}
    \author{  
    Tobias Rücker\\
    \texorpdfstring{\href{mailto:tobias.ruecker@tu-dortmund.de}{tobias.ruecker@tu-dortmund.de}
    \and}{,} 
    Paul Störbrock\\
    \texorpdfstring{\href{mailto:paul.stoerbrock@tu-dortmund.de}{paul.stoerbrock@tu-dortmund.de}}{}
    }
    \date{Durchführung: 03.12.2019, Abgabe: 10.12.2019\vspace{-4ex}}
\maketitle
\center{\Large Versuchsgruppe: \textbf{42}}
    
    \begin{abstract}
    \centering
        \textbf{Ziel:} 
    \end{abstract}

\newpage
\tableofcontents
\newpage

% Theorie %%%%%%%%%%%%%%%%%%%%%%%%%%%%%%%%%%%%%%%%%%%%%%%%%%%%%%%%%%%%%%%%%%%%%%%%%%%%%%%%%%%%%%%%%%%%%%%%%

\section{Theorie}\justifying

% Fehlerrechnung %%%%%%%%%%%%%%%%%%%%%%%%%%%%%%%%%%%%%%%%%%%%%%%%%%%%%%%%%%%%%%%%%%%%%%%%%%%%%%%%%%%%%%%%%%%%%%%%%%%

\section{Fehlerrechnung}\justifying
Für eventuell anfallende Fehler in der Auswertung werden folgende Formeln verwendet:
\begin{subequations}
\begin{align}
\intertext{Für den Mittelwert wird die Formel
}
    \overline{x} &= \frac{1}{N}\sum_{i=1}^{N} x_i \label{eq:a}.
\intertext{verwendet. Der Fehler des Mittelwerts wird mit der Formel:
}
    \Delta\overline{x} &= \frac{1}{\sqrt{N}} \sqrt{\frac{1}{1-N} \sum_{i=1}^{N} (x_i - \overline{x})^2} \label{eq:b},
\intertext{berechnet, und die Gaußsche Fehlerfortpflanzung wird berechnet wie folgt:
}
    \Delta f &= \sqrt{\sum_{i=1}^{N} \left( \frac{\delta f}{\delta x_i} \right)^2 \cdot (\Delta x_i)^2} \label{eq:c}
\intertext{Die für eine lineare Regression benötigte Ausgleichsgerade wird mit den folgenden Parametern berechnet:
}
    y &= m \cdot x + b \label{eq:d} \\ 
    m &= \frac{\overline{xy} - \overline{x} \cdot \overline{y}}{\overline{x^2} - {\overline{x}}^2} \label{eq:e}\\
    b &= \frac{\overline{y} \cdot \overline{x^2} - \overline{xy} \cdot \overline{x}}{\overline{x^2} - {\overline{x}}^2} \label{eq:f}
\end{align}
\end{subequations}
\newpage

% Versuchsaufbau/Versuchsdurchführung %%%%%%%%%%%%%%%%%%%%%%%%%%%%%%%%%%%%%%%%%%%%%%%%%%%%%%%%%%%%%%%%%%%%%%%%%%%%%%%%%%%%%%%%%%%%%%%%%%%%%%%%%%%%%%%%%%%%%%%%%%%%%%%%%%%%%%%%%%%%%%%%%%%%%%

\section{Versuchsaufbau/Durchführung}\justifying

\flushleft{Benötigt\;}\justifying werden: \textit{ Eine ca. $\SI{60}{\centi\meter}$ lange, $\SI{530}{\gram}$ schwere, rechtreckige Kupfterstange, eine ca. $\SI{60}{\centi\meter}$ lange, runde
Aluminiumstange, eine Aperatur zur Biegung der Stäbe mit zwei Fixpunkten, zwei schiebbare Messuhren im $\SI{}{\micro\meter}$-Bereich, eine
$\SI{55}{\centi\meter}$ lange Skala für die Messuhren, eine Aufhängung ($\SI{19}{\gram}$), acht Gewichte ($4x$ ca. $\SI{1160}{\gram}$, 
$2x$ ca. $\SI{500}{\gram}$, $1x$ ca. $\SI{225}{\gram}$), ein Maßband, eine Waage.
}
\flushleft{Zuerst\;}\justifying wird die Kupferstange an einem der Fixpunkte eingespannt. Die Messuhren werden über der Stange auf der Skala platziert. Nun werden mit 
einer Messuhr 19 Messwerte der Durchbiegung in einem Abstand von $\SI{2.5}{\centi\meter}$ (von $\SI{2.5}{\centi\meter}$ bis $\SI{47.5}{\centi\meter}$) 
bestimmt. Nach den Messungen ohne Gewicht, werden zwei $\SI{500}{\gram}$ Gewichte an das lose Ende gehangen und die Messungen wiederholt. Anschließend
werden die Gewichte abgenommen und die Kupferstange an beiden Enden fixiert. Nun wird die Durchbiegung an 14 Stellen mit einem Abstand von 
$\SI{3}{\centi\meter}$ gemessen. Danach werden vier $\SI{1160}{\gram}$ Gewichte mittig an die Stange gehangen und die Messungen wiederholt. 

\flushleft{Der\;}\justifying obige Prozess wird für die Runde Aluminiumstange wiederholt. Zu Beginn wird die Aluminiumstange einseitig eingespannt. Die Durchbiegung
wird ohne Gewichte 19-mal in einem Abstand von $\SI{2.5}{\centi\meter}$ gemessen. Dann wird ein $\SI{500}{\gram}$ Gewicht an das lose Ende 
gehangen und die Messung wiederholt. Anschließend wird das Gewicht abgenommen und die Stange an beiden Seiten fixiert. Abschließend werden 
14 Messungen ohne Gewichte und 14 Messungen mit zwei $\SI{500}{\gram}$ Gewichten und einem $\SI{230}{\gram}$ Gewicht durchgeführt.

% Auswertung %%%%%%%%%%%%%%%%%%%%%%%%%%%%%%%%%%%%%%%%%%%%%%%%%%%%%%%%%%%%%%%%%%%%%%%%%%%%%%%%%%%%%%%%%%%%%%%%%%%%%%%%%%%%%%%%%%%%%%%%%%%%%%%%%%%%%%%%%%%

\section{Auswertung}\justifying

\begin{table}[H]
    \centering
    \input{Cu_ein.tex}
    \caption{Testtabelle}
    \label{tab:1}
\end{table}

% Diskussion %%%%%%%%%%%%%%%%%%%%%%%%%%%%%%%%%%%%%%%%%%%%%%%%%%%%%%%%%%%%%%%%%%%%%%%%%%%%%%%%%%%%%%%%%%%%%%%%%%%%%%%%%%%%%%%%%%%%%%%%%%%%%%%%%%%%%%%%%%%

\section{Diskussion}\justifying

\newpage

\printbibliography
\end{document}