\documentclass[
  captions=tableheading,
  bibliography=totoc, 
  titepage=firstiscover,
]{scrartcl}

\usepackage{blindtext} %neuer input

\usepackage{longtable} % Tabellen über mehrere Seiten

\usepackage[utf8]{inputenc} %neuer input

\usepackage{scrhack}

\usepackage[aux]{rerunfilecheck} %Warnung falls nochmal kompiliert werden muss

\usepackage{fontspec} %Fonteinstellungen

\recalctypearea{}

\usepackage[main=ngerman]{babel} %deutsche Spracheinstellung

\usepackage{ragged2e} %neuer input

\usepackage{amsmath, nccmath}

\usepackage{amssymb} %viele mathe Symbole

\usepackage{mathtools} %Erweiterungen für amsmath


\DeclarePairedDelimiter{\abs}{\lvert}{\rvert}
\DeclarePairedDelimiter{\norm}{\lVert}{\rVert}

\DeclarePairedDelimiter{\bra}{\langle}{\rvert}
\DeclarePairedDelimiter{\ket}{\lvert}{\rangle}

\DeclarePairedDelimiterX{\braket}[2]{\langle}{\rangle}{
#1 \delimsize| #2
}

\NewDocumentCommand \dif {m}
{
\mathinner{\symup{d} #1}
}


\usepackage[
  math-style=ISO,
  bold-style=ISO,
  sans-style=italic,
  nabla=upright,
  partial=upright,
  warnings-off={
    mathtools-colon,
    mathtools-overbracket,
  },
]{unicode-math}

\setmathfont{Latin Modern Math}
\setmathfont{XITS Math}[range={scr, bfscr}]
\setmathfont{XITS Math}[range={cal, bfcal}, StylisticSet=1]


\usepackage[
  locale=DE,
  separate-uncertainty=true,
  per-mode=reciprocal,
  output-decimal-marker={,},
]{siunitx}

\usepackage[autostyle]{csquotes} %richtige Anführungszeichen

\usepackage{xfrac}

\usepackage{float}

\floatplacement{figure}{htbp}

\floatplacement{table}{htbp}

\usepackage[ %floats innerhalb einer section halten
  section,   %floats innerhalb er section halten
  below,     %unterhalb der Section aber auf der selben Seite ist ok
]{placeins}

\usepackage[
  labelfont=bf,
  font=small,
  width=0.9\textwidth,
]{caption}

\usepackage{subcaption} %subfigure, subtable, subref

\usepackage{graphicx}

\usepackage{grffile}

\usepackage{booktabs}

\usepackage{microtype} %Verbesserungen am Schriftbild

\usepackage[
backend=biber,
]{biblatex}

\addbibresource{../lit.bib}

\usepackage[ %Hyperlinks im Dokument
  german,
  unicode,
  pdfusetitle,
  pdfcreator={},
  pdfproducer={},
]{hyperref}

\usepackage{bookmark}

\usepackage[shortcuts]{extdash}

%\usepackage{warpcol}

\usepackage{tikz}

\newcommand*\circled[1]{\tikz[baseline=(char.base)]{
            \node[shape=circle,draw,inner sep=2pt] (char) {#1};}}

\begin{document}
    \title{V602: Röntgenemission und -absorption}
    \author{  
    Paul Störbrock\\
    \texorpdfstring{\href{mailto:paul.stoerbrock@tu-dortmund.de}{paul.stoerbrock@tu-dortmund.de}}{}
    }
    \date{Abgabe: 19.05.2020\vspace{-4ex}}
\maketitle
    
\newpage
\tableofcontents
\newpage

\setcounter{page}{1}

\section{Ziel}

\section{Theorie}

\section{Versuchsaufbau und -durchfürung}

\section{Auswertung}

    \begin{figure}[H]
        \centering
        \includegraphics[width=\textwidth]{build/plotCu.pdf}
        \label{fig:}
    \end{figure}

\section{Diskussion}

\newpage
\section{Literatur}

\newpage
\section{Appendix}

    \input{table_Cu.tex}

    \begin{table}[H]
        \centering
        \caption{Bragg}
        \input{table_bragg.tex}
        \label{tab:2}
    \end{table}

    \begin{table}[H]
        \centering
        \caption{Brom}
        \input{table_brom.tex}
        \label{tab:3}
    \end{table}
    
    \begin{table}[H]
        \centering
        \caption{Gallium}
        \input{table_gallium.tex}
        \label{tab:4}
    \end{table}
    
    \begin{table}[H]
    \centering
        \begin{subtable}{.49\textwidth}
        \centering
        \caption{Rubidium}
        \input{table_rub.tex}
        \label{tab:5a}
        \end{subtable}    
        \begin{subtable}{.49\textwidth}
        \centering
        \caption{Strontium}
        \input{table_stron.tex}
        \label{tab:5b}
        \end{subtable}
        \begin{subtable}{.49\textwidth}
        \centering
        \caption{Zink}
        \input{table_zink.tex}
        \label{tab:5c}
        \end{subtable}    
        \begin{subtable}{.49\textwidth}
        \centering
        \caption{Zirkonium}
        \input{table_zirk.tex}
        \label{tab:5d}
        \end{subtable}
    \end{table}

\end{document}