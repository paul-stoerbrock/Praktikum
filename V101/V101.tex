\documentclass[
  captions=tableheading,
  bibliography=totoc, 
  titepage=firstiscover,
]{scrartcl}

\usepackage{blindtext} %neuer input

\usepackage{longtable} % Tabellen über mehrere Seiten

\usepackage[utf8]{inputenc} %neuer input

\usepackage{scrhack}

\usepackage[aux]{rerunfilecheck} %Warnung falls nochmal kompiliert werden muss

\usepackage{fontspec} %Fonteinstellungen

\recalctypearea{}

\usepackage[main=ngerman]{babel} %deutsche Spracheinstellung

\usepackage{ragged2e} %neuer input

\usepackage{amsmath, nccmath}

\usepackage{amssymb} %viele mathe Symbole

\usepackage{mathtools} %Erweiterungen für amsmath

\usepackage{MnSymbol}


\DeclarePairedDelimiter{\abs}{\lvert}{\rvert}
\DeclarePairedDelimiter{\norm}{\lVert}{\rVert}

\DeclarePairedDelimiter{\bra}{\langle}{\rvert}
\DeclarePairedDelimiter{\ket}{\lvert}{\rangle}

\DeclarePairedDelimiterX{\braket}[2]{\langle}{\rangle}{
#1 \delimsize| #2
}

\usepackage{expl3}
\usepackage{xparse}
\NewDocumentCommand \dif {m}
{
\mathinner{\symup{d} #1}
}

\NewDocumentCommand \del {mm}
{
    \mathinner{\frac{\partial #1}{\partial #2}}
}
\NewDocumentCommand \deln {mmm}
{
    \mathinner{\frac{\partial^#3 #1}{\partial #2 ^#3}}
}
\ExplSyntaxOff

\usepackage[
  math-style=ISO,
  bold-style=ISO,
  sans-style=italic,
  nabla=upright,
  partial=upright,
  warnings-off={
    mathtools-colon,
    mathtools-overbracket,
  },
]{unicode-math}

\setmathfont{Latin Modern Math}
\setmathfont{XITS Math}[range={scr, bfscr}]
\setmathfont{XITS Math}[range={cal, bfcal}, StylisticSet=1]

\usepackage[
version=4,
math-greek=default,
text-greek=default,
]{mhchem}

\usepackage[
  locale=DE,
  separate-uncertainty=true,
  per-mode=reciprocal,
  output-decimal-marker={,},
]{siunitx}

\usepackage[autostyle]{csquotes} %richtige Anführungszeichen

\usepackage{xfrac}

\usepackage{float}

\floatplacement{figure}{htbp}

\floatplacement{table}{htbp}

\usepackage[ %floats innerhalb einer section halten
  section,   %floats innerhalb er section halten
  below,     %unterhalb der Section aber auf der selben Seite ist ok
]{placeins}

\usepackage[
  labelfont=bf,
  font=small,
  width=0.9\textwidth,
]{caption}

\usepackage{subcaption} %subfigure, subtable, subref

\usepackage{graphicx}

\usepackage{ulem}

\usepackage{color}

\usepackage{grffile}

\usepackage{booktabs}

\sisetup{separate-uncertainty=true}

\usepackage{microtype} %Verbesserungen am Schriftbild

\usepackage[
backend=biber,
]{biblatex}

\addbibresource{../lit.bib}

\usepackage[ %Hyperlinks im Dokument
  german,
  unicode,
  pdfusetitle,
  pdfcreator={},
  pdfproducer={},
]{hyperref}

\usepackage{bookmark}

\usepackage[shortcuts]{extdash}



\begin{document}
    \title{V101 Das Trägheitsmoment}
    \author{  
    Tobias Rücker\\
    \texorpdfstring{\href{mailto:tobias.ruecker@tu-dortmund.de}{tobias.ruecker@tu-dortmund.de}
    \and}{,} 
    Paul Störbrock\\
    \texorpdfstring{\href{mailto:paul.stoerbrock@tu-dortmund.de}{paul.stoerbrock@tu-dortmund.de}}{}
    }
    \date{Durchführung: 03.12.2019, Abgabe: 10.12.2019\vspace{-4ex}}
\maketitle
\center{\Large Versuchsgruppe: \textbf{42}}
    
    \begin{abstract}
    \centering
        \textbf{Ziel: Messung des Trägheitsmoment verschiedener Körper und Nachweis des Satzes von Steiner} 
    \end{abstract}

\newpage
\tableofcontents
\newpage

% Theorie %%%%%%%%%%%%%%%%%%%%%%%%%%%%%%%%%%%%%%%%%%%%%%%%%%%%%%%%%%%%%%%%%%%%%%%%%%%%%%%%%%%%%%%%%%%%%%%%%
\section{Theorie}\justifying
\begin{align}
    \intertext{Rotationsbewegungen beschreiben Bewegungen bei denen sich eine Massenverteilung
    um eine Rotationsachse drehen. Dabei werden diese Bewegungen hauptsächlich durch dre Größen
    beschrieben: dem Trägheitsmoment $I$, dem Drehmoment $M$ und der Winkelbeschleunigung $\dot{\omega}$.
    Das Trägheitsmoment beschreibt dabei eine skalare Größe, welche ein Maß der Trägheit einer Massenverteilung
    gegenüber Rotatonen darstellen. Es bildet damit ein Aquivalent zur Masse in Translationsbewegungen.
    Ein Trägheitsmoment hängt dabei explizit von dem Abstand und der Wahl der Rotationsachse ab.
    Für eine Punktmasse zum Beispiel lässt sich das Trägheitsmoment durch folgende Formel 
    beschreiben \cite{V101} :
    }
    I = m  r^2 \label{eq:1}
    \intertext{Für ausgedehnte Massekörper wird diese Formel über infinitesimal kleine Massestücke aufintegriert \cite{V101} :
    }
    I = \int r^2 \symup{d} \, m \label{eq:2}
    \intertext{Für einfache Körper lassen sich diese Integrale einfach lösen. 
    Einige Beispiele für Trägheitsmomente einfacher Körper sind dabei \cite{V101} :
    }
    I_{\text{Kugel}}= \frac{2}{5} m R^2 \label{eq:3}
    \intertext{-das Trägheitsmoment einer homogenen Vollkugel durch seinen Mittelpunkt}
    I_{\text{Zylinder,v}}= \frac{mR^2}{2} \label{eq:4}
    \intertext{-das Trägheitsmoment eines homogenen Vollzylinders mit der Drehachse durch den  Mittelpunkt des Bodens}
    I_{\text{Zylinder,h}}= m \left( \frac{R^2}{4}+\frac{h^2}{12} \right) \label{eq:5}
    \intertext{-das Trägheitsmoment eines homogenen Vollzylinders deren Drehachse senkrecht zum Mantel durch den
    Schwerpunkt verläuft.
    Bei komplexeren Körper wird versucht diesen durch eine Zusammensetzung einfacher Körper zu bechreiben.
    Verläuft die Drehachse parallel zu zu einer Achse durch den Schwerpunkt, wird das Trägheitsmoment
    mithilfe des Steinerschen Satzes berechnet \cite{V101} :
    }
    I=I_s+m \cdot a^2 \label{eq.6}
    \intertext{$I_s$ beschreibt dabei das Trägheitsmoment bezüglich der parallelen Drehachse durch den Schwerpunkt.
    die skalare Größe a stellt dabei den Abstand der beiden parallelen Achsen.
    Das Drehmoment ist definiert als die Kraft, welche an einem drehenden Körper ansetzt.
    Beschreiben lässt sich dieser durch \cite{V101} :
    }
    \vec{M} = \vec{F} \times \vec{r} \label{eq:7}
    \intertext{Liegt nun ein oszillierendes System vor, welches um einen Winkel $\varphi$
    ausgelenkt, so wirkt eine Feder eine rücktreibende Kraft der Bewegung entgegen.
    Die Periodendauer der Schwingung kann dabei beschrieben werden durch \cite{V101} :
    }
    T=2 \pi \sqrt{\frac{I}{D}} \label{eq:8}
    \intertext{Wobei das Dremoment und die Winkelrichtgröße in folgender Beziehung
    stehen \cite{V101} :
    }
    M = D \cdot \varphi \label{eq:9}
\end{align}

% Fehlerrechnung %%%%%%%%%%%%%%%%%%%%%%%%%%%%%%%%%%%%%%%%%%%%%%%%%%%%%%%%%%%%%%%%%%%%%%%%%%%%%%%%%%%%%%%%%%%%%%%%%%%
\section{Fehlerrechnung}\justifying
Für die Auswertung werden im folgenden diese Formeln zur Bestimmung der Fehler verwendet:
\begin{subequations}
\begin{align}
\intertext{Den Mittelwert wird berechnet mit}
     \overline{x} &= \frac{1}{N}\sum_{i=1}^{N} x_i \label{eq:10a}.
\intertext{Der Fehler des Mittelwerts wird berechnet mit}
        \Delta\overline{x} &= \frac{1}{\sqrt{N}} \sqrt{\frac{1}{1-N} \sum_{i=1}^{N} (x_i - \overline{x})^2} \label{eq:10b},
\intertext{Die Gaußsche Fehlerfortpflanzung wird mit der folgenden Formel berechnet} \text{und}
        \Delta f &= \sqrt{\sum_{i=1}^{N} \left( \frac{\delta f}{\delta x_i} \right)^2 \cdot (\Delta x_i)^2} \label{eq:10c}
        \intertext{verwendet.}
\intertext{Zur Berechnung von Ausgleichsgeraden werden dabei}
        y &= m \cdot x + b \label{eq:10d} \\ 
        m &= \frac{\overline{xy} - \overline{x} \cdot \overline{y}}{\overline{x^2} - {\overline{x}}^2} \label{eq:10e} \qquad \qquad\\
        b = \frac{\overline{y} \cdot \overline{x^2} - \overline{xy} \cdot \overline{x}}{\overline{x^2} - {\overline{x}}^2} \label{eq:10f}
\end{align}
\end{subequations}


% Versuchsaufbau %%%%%%%%%%%%%%%%%%%%%%%%%%%%%%%%%%%%%%%%%%%%%%%%%%%%%%%%%%%%%%%%%%%%%%%%%%%%%%%%%%%%%%%%%%%%%%%%%%%%%%%%%%%%%%%%%%%%%%%%%%%%%%%%%%%%%%%%%%%
\section{Versuchsaufbau}\justifying

Benötigt werden: \textit{Eine Drehachse die durch eine Spiralfeder mit einem festen Rahmen verbunden ist, eine Stabachse der Länge 62cm, zwei
identische zylindrische Gewichte (261.5g), eine Kugel mit Aufsatz (m=1168.5g, r=7.3cm), einen Zylinder mit Aufsatz (m=1525.5g, h=13,95cm, d=8cm), 
eine Holzpuppe mit Aufsatz ($\rho$=), ein Newtonmeter (von 0.1N bis 1N), ein Ma\ss band, eine Schieblehre, eine Stopuhr}


% Versuchsdurchführung %%%%%%%%%%%%%%%%%%%%%%%%%%%%%%%%%%%%%%%%%%%%%%%%%%%%%%%%%%%%%%%%%%%%%%%%%%%%%%%%%%%%%%%%%%%%%%%%%%%%%%%%%%%%%%%%%%%%%%%%%%%%%%%%%%%%%%%%%%%
\section{Versuchsdurchführung}\justifying

% Auswertung %%%%%%%%%%%%%%%%%%%%%%%%%%%%%%%%%%%%%%%%%%%%%%%%%%%%%%%%%%%%%%%%%%%%%%%%%%%%%%%%%%%%%%%%%%%%%%%%%%%%%%%%%%%%%%%%%%%%%%%%%%%%%%%%%%%%%%%%%%%
\section{Auswertung}\justifying

\subsection{Drehachse}\justifying

\subsection{Zylinder}\justifying

\subsection{Kugel}\justifying

\subsection{Puppe}\justifying

% Diskussion %%%%%%%%%%%%%%%%%%%%%%%%%%%%%%%%%%%%%%%%%%%%%%%%%%%%%%%%%%%%%%%%%%%%%%%%%%%%%%%%%%%%%%%%%%%%%%%%%%%%%%%%%%%%%%%%%%%%%%%%%%%%%%%%%%%%%%%%%%%
\section{Diskussion}\justifying

\newpage

\printbibliography
\end{document}