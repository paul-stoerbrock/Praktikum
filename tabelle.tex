\documentclass[
  captions=tableheading,
]{scrartcl}

\usepackage{scrhack}

\usepackage[aux]{rerunfilecheck}

\usepackage{fontspec}

\usepackage[main=ngerman]{babel}

\usepackage{amsmath}
\usepackage{amssymb}
\usepackage{mathtools}


\usepackage[
  math-style=ISO,
  bold-style=ISO,
  sans-style=italic,
  nabla=upright,
  partial=upright,
]{unicode-math}

\usepackage[
  locale=DE,
  separate-uncertainty=true,
  per-mode=symbol-or-fraction,
]{siunitx}

\usepackage{booktabs}

\usepackage[unicode]{hyperref}
\usepackage{bookmark}

\begin{document}

\begin{table}
  \centering
  \caption{
    Eine Tabelle mit Messdaten.
    Wir werden später lernen, wie man sie zentriert.
  }
  \input{loesung-table.tex}
\end{table}

\begin{table}
  \centering
  \caption{Eine Tabelle mit Messwert und Fehler.}
  \begin{tabular}{
      S[table-format=2.2]
      @{${}\pm{}$}
      S[table-format=1.2]
  }
    \toprule
    \multicolumn{2}{c}{$x \mathbin{/} \si{\metre}$} \\
    \midrule
     9.29 & 0.79 \\
     7.6  & 1.7  \\
    16.4  & 6.5  \\
    10.03 & 0.51 \\
     9.0  & 1.7  \\
    10.5  & 1.1  \\
    10.49 & 0.29 \\
    10.5  & 1.6  \\
     9.9  & 1.2  \\
    10.64 & 0.80 \\
     9.3  & 1.0  \\
     9.28 & 0.88 \\
    10.96 & 0.69 \\
    10.48 & 0.72 \\
     9.8  & 1.4  \\
     9.58 & 0.33 \\
    10.2  & 2.1  \\
    10.31 & 0.91 \\
    10.53 & 0.42 \\
     8.5  & 2.0  \\
    \bottomrule
  \end{tabular}
\end{table}

\end{document}
