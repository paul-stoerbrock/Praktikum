 
\documentclass[
  captions=tableheading,
  bibliography=totoc, 
  titepage=firstiscover,
]{scrartcl}

\usepackage{blindtext} %neuer input

\usepackage{longtable} % Tabellen über mehrere Seiten

\usepackage[utf8]{inputenc} %neuer input

\usepackage{scrhack}

\usepackage[aux]{rerunfilecheck} %Warnung falls nochmal kompiliert werden muss

\usepackage{fontspec} %Fonteinstellungen

\recalctypearea{}

\usepackage[main=ngerman]{babel} %deutsche Spracheinstellung

\usepackage{ragged2e} %neuer input

\usepackage{amsmath, nccmath}

\usepackage{amssymb} %viele mathe Symbole

\usepackage{mathtools} %Erweiterungen für amsmath


\DeclarePairedDelimiter{\abs}{\lvert}{\rvert}
\DeclarePairedDelimiter{\norm}{\lVert}{\rVert}

\DeclarePairedDelimiter{\bra}{\langle}{\rvert}
\DeclarePairedDelimiter{\ket}{\lvert}{\rangle}

\DeclarePairedDelimiterX{\braket}[2]{\langle}{\rangle}{
#1 \delimsize| #2
}

\NewDocumentCommand \dif {m}
{
\mathinner{\symup{d} #1}
}


\usepackage[
  math-style=ISO,
  bold-style=ISO,
  sans-style=italic,
  nabla=upright,
  partial=upright,
  warnings-off={
    mathtools-colon,
    mathtools-overbracket,
  },
]{unicode-math}

\setmathfont{Latin Modern Math}
\setmathfont{XITS Math}[range={scr, bfscr}]
\setmathfont{XITS Math}[range={cal, bfcal}, StylisticSet=1]


\usepackage[
  locale=DE,
  separate-uncertainty=true,
  per-mode=reciprocal,
  output-decimal-marker={,},
]{siunitx}

\usepackage[autostyle]{csquotes} %richtige Anführungszeichen

\usepackage{xfrac}

\usepackage{float}

\floatplacement{figure}{htbp}

\floatplacement{table}{htbp}

\usepackage[ %floats innerhalb einer section halten
  section,   %floats innerhalb er section halten
  below,     %unterhalb der Section aber auf der selben Seite ist ok
]{placeins}

\usepackage[
  labelfont=bf,
  font=small,
  width=0.9\textwidth,
]{caption}

\usepackage{subcaption} %subfigure, subtable, subref

\usepackage{graphicx}

\usepackage{grffile}

\usepackage{booktabs}

\usepackage{microtype} %Verbesserungen am Schriftbild

\usepackage[
backend=biber,
]{biblatex}

\addbibresource{../lit.bib}

\usepackage[ %Hyperlinks im Dokument
  german,
  unicode,
  pdfusetitle,
  pdfcreator={},
  pdfproducer={},
]{hyperref}

\usepackage{bookmark}

\usepackage[shortcuts]{extdash}

%\usepackage{warpcol}

\begin{document}
    \title{V803 Das Hooksche Gesetz}
    \author{  
    Tobias Rücker\\
    \texorpdfstring{\href{mailto:tobias.ruecker@tu-dortmund.de}{tobias.ruecker@tu-dortmund.de}
    \and}{,} 
    Paul Störbrock\\
    \texorpdfstring{\href{mailto:paul.stoerbrock@tu-dortmund.de}{paul.stoerbrock@tu-dortmund.de}}{}
    }
    \date{Durchführung: 01.11.2019, Abgabe: 05.11.2019\vspace{-4ex}}
\maketitle
\center{\Large Versuchsgruppe: \textbf{42}}
    
    \begin{abstract}
        Ziel: Beweis des hookschen Gesetzes über Bestimmung der Federkonstante. 
    \end{abstract}
    
\newpage

\flushleft
\section{Versuchsaufbau:}
\section{Versuchsdurchführung:}
   
    \quad Dies ist ein Beispieltext der wunderschön im Dokument ausschauen soll, aber vielleicht ein error aufrufen wird. zehn zahme ziegen ziehen zehn cent zucker.
\section{Messdaten:}   
\begin{table}
    \centering
    \label{tab:data}
    \input{table.tex}   
\end{table}
    \subsection{Mittelwert:}
    \begin{equation}
        \overline{x} = \frac{1}{N}\sum\limits_{i=1}^{N} x_i
    \end{equation}
    Mittelwert\:$\overline{x}$\:=\:\input{meanD.tex}

    \subsection{Lineare Ausgleichsrechnung:}
    \begin{align}\label{eq.linreg}
        m &= \frac{\overline{xy}-\overline{x}* \overline{y}}{\overline{x^2}-\overline{x}^2} 
        &b = \frac{\overline{y}* \overline{x^2}-\overline{xy}* \overline{x}}{\overline{x^2}- \overline{x}^2}
    \end{align}
    Federkonstante\:D\:=\:\input{linregD.tex}

    
    
\end{document}
