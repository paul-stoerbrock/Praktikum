 
\documentclass[
  captions=tableheading,
  bibliography=totoc, 
  titepage=firstiscover,
]{scrartcl}

\usepackage{blindtext} %neuer input

\usepackage{longtable} % Tabellen über mehrere Seiten

\usepackage[utf8]{inputenc} %neuer input

\usepackage{scrhack}

\usepackage[aux]{rerunfilecheck} %Warnung falls nochmal kompiliert werden muss

\usepackage{fontspec} %Fonteinstellungen

\recalctypearea{}

\usepackage[main=ngerman]{babel} %deutsche Spracheinstellung

\usepackage{ragged2e} %neuer input

\usepackage{amsmath, nccmath}

\usepackage{amssymb} %viele mathe Symbole

\usepackage{mathtools} %Erweiterungen für amsmath


\DeclarePairedDelimiter{\abs}{\lvert}{\rvert}
\DeclarePairedDelimiter{\norm}{\lVert}{\rVert}

\DeclarePairedDelimiter{\bra}{\langle}{\rvert}
\DeclarePairedDelimiter{\ket}{\lvert}{\rangle}

\DeclarePairedDelimiterX{\braket}[2]{\langle}{\rangle}{
#1 \delimsize| #2
}

\NewDocumentCommand \dif {m}
{
\mathinner{\symup{d} #1}
}


\usepackage[
  math-style=ISO,
  bold-style=ISO,
  sans-style=italic,
  nabla=upright,
  partial=upright,
  warnings-off={
    mathtools-colon,
    mathtools-overbracket,
  },
]{unicode-math}

\setmathfont{Latin Modern Math}
\setmathfont{XITS Math}[range={scr, bfscr}]
\setmathfont{XITS Math}[range={cal, bfcal}, StylisticSet=1]


\usepackage[
  locale=DE,
  separate-uncertainty=true,
  per-mode=reciprocal,
  output-decimal-marker={,},
]{siunitx}

\usepackage[autostyle]{csquotes} %richtige Anführungszeichen

\usepackage{xfrac}

\usepackage{float}

\floatplacement{figure}{htbp}

\floatplacement{table}{htbp}

\usepackage[ %floats innerhalb einer section halten
  section,   %floats innerhalb er section halten
  below,     %unterhalb der Section aber auf der selben Seite ist ok
]{placeins}

\usepackage[
  labelfont=bf,
  font=small,
  width=0.9\textwidth,
]{caption}

\usepackage{subcaption} %subfigure, subtable, subref

\usepackage{graphicx}

\usepackage{grffile}

\usepackage{booktabs}

\usepackage{microtype} %Verbesserungen am Schriftbild

\usepackage[
backend=biber,
]{biblatex}

\addbibresource{../lit.bib}

\usepackage[ %Hyperlinks im Dokument
  german,
  unicode,
  pdfusetitle,
  pdfcreator={},
  pdfproducer={},
]{hyperref}

\usepackage{bookmark}

\usepackage[shortcuts]{extdash}

%\usepackage{warpcol}

\begin{document}
    \title{V803 Das Hooksche Gesetz}
    \author{  
    Tobias Rücker\\
    \texorpdfstring{\href{mailto:tobias.ruecker@tu-dortmund.de}{tobias.ruecker@tu-dortmund.de}
    \and}{,} 
    Paul Störbrock\\
    \texorpdfstring{\href{mailto:paul.stoerbrock@tu-dortmund.de}{paul.stoerbrock@tu-dortmund.de}}{}
    }
    \date{Durchführung: 01.11.2019, Abgabe: 05.11.2019\vspace{-4ex}}
\maketitle
\center{\Large Versuchsgruppe: \textbf{42}}
    
    \begin{abstract}
        Ziel: Beweis des hookschen Gesetzes über Bestimmung der Federkonstante. 
    \end{abstract}
    
\newpage
\justifying
\section{Versuchsaufbau:}
    Benötigt werden:    \textit{Zwei Stative, eine Feder, ein ca. 1m langes Stück Faden, ein 60cm Lineal, eine Umlenkrolle,
    ein digitales Newton Meter, ein zugehöriger Adapter, ein Auslesegerät (Laptop)} \\
    Die Feder wird an ein Newton Meter gehängt, welches von einem Stativ gehalten wird. An das freie untere Ende der Feder wird eine Schnur
    befestigt, welche über eine Umlenkrolle mit einem Schieber befestigt wird. Der Schieber soll die Auslenkung auf dem waagerechten Lineal 
    anzeigen, welches zwischen die beiden Stative befestigt wird. Das Newton Meter wird an einen Laptop angeschlossen um die entsprechende
    Kraft zur Auslenkung auszugeben.

\section{Versuchsdurchführung:}
    Mithilfe des Schiebers am Lineal wird die Feder auf eine beliebige Entfernung zwischen 0cm und 60cm mit einer Genauigkeit von maximal
    0.5cm ausgelenkt. Es werden für zehn Messungen die Auslenkung und die Kraft bestimmt und in der Tabelle \ref{tab:data} notiert.
    Anschlie\ss end wird mit Hilfe der Formel \ref{eq.mean} der Mittelwert von der Federkonstante  $D$ berechnet.
    \begin{align*}
    \text{Das hooksche Gesetz:\:}D\:=\:F\:\cdot\:\Delta\:x^{-1}\vspace{-4ex}
    \end{align*}

\section{Messdaten:} 
    
\begin{table}
    \centering
    \input{table.tex} 
    \caption{Messdaten}
    \label{tab:data}
\end{table}
    \subsection{Mittelwert:}
    \begin{equation}\label{eq.mean}
        \overline{x} = \frac{1}{N}\sum\limits_{i=1}^{N} x_i
    \end{equation}
    Mittelwert von D: $\overline{x}$\:=\:\input{meanD.tex}

    \subsection{Lineare Ausgleichsrechnung:}
    \begin{align}\label{eq.linreg}
        m &= \frac{\overline{xy}-\overline{x}* \overline{y}}{\overline{x^2}-\overline{x}^2} 
        &b = \frac{\overline{y}* \overline{x^2}-\overline{xy}* \overline{x}}{\overline{x^2}- \overline{x}^2}
    \end{align}
    wobei hier $\; x = \Delta x \text{,} \; y=F \;\;\text{und}\; m = D \\$
    Federkonstante\:D\:=\:\input{linregD.tex}

    
    
\end{document}
