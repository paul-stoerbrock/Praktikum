\documentclass[
  captions=tableheading,
  bibliography=totoc, 
  titepage=firstiscover,
]{scrartcl}

\usepackage{blindtext} %neuer input

\usepackage{longtable} % Tabellen über mehrere Seiten

\usepackage[utf8]{inputenc} %neuer input

\usepackage{scrhack}

\usepackage[aux]{rerunfilecheck} %Warnung falls nochmal kompiliert werden muss

\usepackage{fontspec} %Fonteinstellungen

\recalctypearea{}

\usepackage[main=ngerman]{babel} %deutsche Spracheinstellung

\usepackage{ragged2e} %neuer input

\usepackage{amsmath, nccmath}

\usepackage{amssymb} %viele mathe Symbole

\usepackage{mathtools} %Erweiterungen für amsmath

\usepackage{MnSymbol}


\DeclarePairedDelimiter{\abs}{\lvert}{\rvert}
\DeclarePairedDelimiter{\norm}{\lVert}{\rVert}

\DeclarePairedDelimiter{\bra}{\langle}{\rvert}
\DeclarePairedDelimiter{\ket}{\lvert}{\rangle}

\DeclarePairedDelimiterX{\braket}[2]{\langle}{\rangle}{
#1 \delimsize| #2
}

\usepackage{expl3}
\usepackage{xparse}
\NewDocumentCommand \dif {m}
{
\mathinner{\symup{d} #1}
}

\NewDocumentCommand \del {mm}
{
    \mathinner{\frac{\partial #1}{\partial #2}}
}
\NewDocumentCommand \deln {mmm}
{
    \mathinner{\frac{\partial^#3 #1}{\partial #2 ^#3}}
}
\ExplSyntaxOff

\usepackage[
  math-style=ISO,
  bold-style=ISO,
  sans-style=italic,
  nabla=upright,
  partial=upright,
  warnings-off={
    mathtools-colon,
    mathtools-overbracket,
  },
]{unicode-math}

\setmathfont{Latin Modern Math}
\setmathfont{XITS Math}[range={scr, bfscr}]
\setmathfont{XITS Math}[range={cal, bfcal}, StylisticSet=1]

\usepackage[
version=4,
math-greek=default,
text-greek=default,
]{mhchem}

\usepackage[
  locale=DE,
  separate-uncertainty=true,
  per-mode=reciprocal,
  output-decimal-marker={,},
]{siunitx}

\usepackage[autostyle]{csquotes} %richtige Anführungszeichen

\usepackage{xfrac}

\usepackage{float}

\floatplacement{figure}{htbp}

\floatplacement{table}{htbp}

\usepackage[ %floats innerhalb einer section halten
  section,   %floats innerhalb er section halten
  below,     %unterhalb der Section aber auf der selben Seite ist ok
]{placeins}

\usepackage[
  labelfont=bf,
  font=small,
  width=0.9\textwidth,
]{caption}

\usepackage{subcaption} %subfigure, subtable, subref

\usepackage{graphicx}

\usepackage{ulem}

\usepackage{color}

\usepackage{grffile}

\usepackage{booktabs}

\sisetup{separate-uncertainty=true}

\usepackage{microtype} %Verbesserungen am Schriftbild

\usepackage[
backend=biber,
]{biblatex}

\addbibresource{../lit.bib}

\usepackage[ %Hyperlinks im Dokument
  german,
  unicode,
  pdfusetitle,
  pdfcreator={},
  pdfproducer={},
]{hyperref}

\usepackage{bookmark}

\usepackage[shortcuts]{extdash}



\begin{document}
    \title{V354 Gedämpfte und erzwungene Schwingungen}
    \author{  
    Tobias Rücker\\
    \texorpdfstring{\href{mailto:tobias.ruecker@tu-dortmund.de}{tobias.ruecker@tu-dortmund.de}
    \and}{,} 
    Paul Störbrock\\
    \texorpdfstring{\href{mailto:paul.stoerbrock@tu-dortmund.de}{paul.stoerbrock@tu-dortmund.de}}{}
    }
    \date{Durchführung: 19.11.2019, Abgabe: 03.12.2019\vspace{-4ex}}
\maketitle
\center{\Large Versuchsgruppe: \textbf{42}}
    
    \begin{abstract}
    \centering
        \textbf{Ziel:} 
    \end{abstract}

\newpage
\tableofcontents
\newpage

% Theorie %%%%%%%%%%%%%%%%%%%%%%%%%%%%%%%%%%%%%%%%%%%%%%%%%%%%%%%%%%%%%%%%%%%%%%%%%%%%%%%%%%%%%%%%%%%%%%%%%%%%%%%%%%%%%%%%%%%%%%%%%%%%%%%%%%%%%%%%%%%

\section{Theorie}\justifying
\begin{align}
\intertext{Ein Schwingkreis beschreibt einen Schalkreis bestehend aus einem
Kondensator und einer Spule. Die Spule besitzt dabei die Induktivität L und der
Kondensator die Kapazität C. Wird auf diese Schaltung nun eine Spannung $U_0$ 
gegeben, so oszilliert der Schwingkreis und führt eine ungedämpfte Schwingung aus. 
Wird nun ein Widerstand in die Schaltung eingebaut, so verliert das System mit der
Zeit Energie, die in Form von Joulscher Wärme an den Widerstand irreversibel
übertragen wird. 
Dadurch führt das System eine gedämpfte Schwingung aus.
Die zeitliche Veränderung der Stromstärke kann dabei durch folgende Formel
beschrieben werden\cite{V354}:}
\symbffrak{I}(t) &= \symbffrak{U}_1e^ {i\tilde{\omega}_1 t}+\symbffrak{U}_2 e^{i \tilde{\omega}_2 t} \text{, wobei}\\
\tilde{\omega} _{1,2} &= i\frac{R}{2L} \pm \sqrt{\frac{1}{LC}-\frac{R^2}{4L^2}} \label{eq:omega}\\
\intertext{ist. 
Aus diesem Zusammenhang ergeben sich drei verschiedene Fälle.
Der erste Fall beshreibt, dass \cite{V354}
}
\frac{1}{LC} & > \frac{R^2}{4L^2} \label{eq:Fall1}
\intertext{ ist und damit der Wurzelterm von $\tilde{\omega}$ \eqref{eq:omega} komplex wird.
    Die zugehörige Schwingungsgleichung entspricht damit:
}
I(t)=A_0 e^{-2 \pi \mu t} \cos (2 \pi \nu t + \eta) \label{eq:I}
\intertext{
 Diese Gleichung beschreibt eine ungedämpfte Schwingung. Die Amlitude $A_0$ läuft dabei
 exponentiell gegen null.}
\end{align}
\justify
Die Schwingungsdauer lässt sich dabei für diesen Fall schreiben als:

\begin{align}
    T_0 =\frac{2 \pi}{\omega _0}=2 \pi \sqrt{LC} \text{, wobei sich nach einer Zeit $T_{ex}$}\\
    T_{ex}:= \frac{1}{2 \pi \mu}=\frac{2L}{R} \label{eq:Abkling}
\end{align}
\justify
die Spannungsamplitude um den Faktor $\sfrac{1}{e}$ verringert hat.
$T_{ex}$ wird daher auch Ablinkdauer genannt.
\justify
Der zweite Fall
\begin{align}
    \frac{1}{LC} <\frac{R^2}{4L^2} \label{eq:Fall2}
\end{align}
stellt den Kriechfall dar. Bei diesem verschwindet der oszillatorische Teil und
die Wurzel aus Gleichung \eqref{eq:omega} wird positiv.
Die Funktion geht mit maximal einem Nulldurchgang nicht-oszillatorisch gegen null.
\justify
Der dritte Fall stellt einen Spezialfall des zweiten dar.
Bei diesem ist die Bedingung:
\begin{align}
    \frac{1}{LC} = \frac{R^2}{4L^2} \label{eq:Fall3}
\end{align}




% Aufgaben %%%%%%%%%%%%%%%%%%%%%%%%%%%%%%%%%%%%%%%%%%%%%%%%%%%%%%%%%%%%%%%%%%%%%%%%%%%%%%%%%%%%%%%%%%%%%%%%%%%%%%%%%%%%%%%%%%%%%%%%%%%%%%%%%%%%%%%%%%%

\section{Aufgaben}\justifying

% Versuchsaufbau %%%%%%%%%%%%%%%%%%%%%%%%%%%%%%%%%%%%%%%%%%%%%%%%%%%%%%%%%%%%%%%%%%%%%%%%%%%%%%%%%%%%%%%%%%%%%%%%%%%%%%%%%%%%%%%%%%%%%%%%%%%%%%%%%%%%%%%%%%%

\section{Versuchsaufbau}\justifying

% Versuchsdurchführung %%%%%%%%%%%%%%%%%%%%%%%%%%%%%%%%%%%%%%%%%%%%%%%%%%%%%%%%%%%%%%%%%%%%%%%%%%%%%%%%%%%%%%%%%%%%%%%%%%%%%%%%%%%%%%%%%%%%%%%%%%%%%%%%%%%%%%%%%%%

\section{Versuchsdurchführung}\justifying

% Auswertung %%%%%%%%%%%%%%%%%%%%%%%%%%%%%%%%%%%%%%%%%%%%%%%%%%%%%%%%%%%%%%%%%%%%%%%%%%%%%%%%%%%%%%%%%%%%%%%%%%%%%%%%%%%%%%%%%%%%%%%%%%%%%%%%%%%%%%%%%%%


\section{Auswertung}\justifying

  \begin{table}[H]
        \centering
        \caption{Messdaten von c) und d)}
        \input{table_c.tex} 
        \label{tab:datac}
  \end{table}

% Diskussion %%%%%%%%%%%%%%%%%%%%%%%%%%%%%%%%%%%%%%%%%%%%%%%%%%%%%%%%%%%%%%%%%%%%%%%%%%%%%%%%%%%%%%%%%%%%%%%%%%%%%%%%%%%%%%%%%%%%%%%%%%%%%%%%%%%%%%%%%%%

\section{Diskussion}\justifying

\newpage
\nocite{V354}
\nocite{V353}
\printbibliography
\end{document}




